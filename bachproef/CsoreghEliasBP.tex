%===============================================================================
% LaTeX sjabloon voor de bachelorproef toegepaste informatica aan HOGENT
% Meer info op https://github.com/HoGentTIN/latex-hogent-report
%===============================================================================

\documentclass[english,dit,thesis]{hogentreport}

% TODO:
% - If necessary, replace the option `dit`' with your own department!
%   Valid entries are dbo, dbt, dgz, dit, dlo, dog, dsa, soa
% - If you write your thesis in English (remark: only possible after getting
%   explicit approval!), remove the option "dutch," or replace with "english".

\usepackage{lipsum} % For blind text, can be removed after adding actual content

%% Pictures to include in the text can be put in the graphics/ folder
\graphicspath{{../graphics/}}

%% For source code highlighting, requires pygments to be installed
%% Compile with the -shell-escape flag!
%% \usepackage[chapter]{minted}
%% If you compile with the make_thesis.{bat,sh} script, use the following
%% import instead:
\usepackage[chapter]{minted}
\usemintedstyle{solarized-light}

%% Formatting for minted environments.
\setminted{%
    autogobble,
    frame=lines,
    breaklines,
    linenos,
    tabsize=4
}

%% Ensure the list of listings is in the table of contents
\renewcommand\listoflistingscaption{%
    \IfLanguageName{dutch}{Lijst van codefragmenten}{List of listings}
}
\renewcommand\listingscaption{%
    \IfLanguageName{dutch}{Codefragment}{Listing}
}
\renewcommand*\listoflistings{%
    \cleardoublepage\phantomsection\addcontentsline{toc}{chapter}{\listoflistingscaption}%
    \listof{listing}{\listoflistingscaption}%
}

% Other packages not already included can be imported here

%%---------- Document metadata -------------------------------------------------
% TODO: Replace this with your own information
\author{Elias Csoregh}
\supervisor{\IfLanguageName{dutch}{Mevr. M. Van Audenrode}{Ms. M. Van Audenrode}}
\cosupervisor{\IfLanguageName{dutch}{Dhr. A. Leflon}{Mr. A. Leflon}}

% \supervisor{Mevr. M. Van Audenrode}
% \cosupervisor{Dhr. A. Leflon}
\title[]%
    {\IfLanguageName{dutch}{Zelfgehoste LLM’s voor Beveiligd Coderen in Bedrijfsomgevingen}{Self-Hosted LLMs for Secure Coding in Enterprise Environments}}
\academicyear{\advance\year by -1 \the\year--\advance\year by 1 \the\year}
\examperiod{1}
\degreesought{\IfLanguageName{dutch}{Professionele bachelor in de toegepaste informatica}{Bachelor of applied computer science}}
\partialthesis{false} %% To display 'in partial fulfilment'
%\institution{Internshipcompany BVBA.}

%% Add global exceptions to the hyphenation here
\hyphenation{back-slash}

%% The bibliography (style and settings are  found in hogentthesis.cls)
\addbibresource{bachproef.bib}            %% Bibliography file
\addbibresource{../voorstel/voorstel.bib} %% Bibliography research proposal
\defbibheading{bibempty}{}

%% Prevent empty pages for right-handed chapter starts in twoside mode
\renewcommand{\cleardoublepage}{\clearpage}

\renewcommand{\arraystretch}{1.2}

\usepackage{tikz}
\usetikzlibrary{shapes, shadows.blur, shapes.multipart}
\usepackage[edges]{forest}
\definecolor{folderbg}{RGB}{124,166,198}
\definecolor{folderborder}{RGB}{110,144,169}
\usetikzlibrary{shapes, backgrounds, fit, calc, positioning, shadows.blur, trees}

\tikzset{
  folder/.pic={
    \filldraw[draw=black, top color=blue!20, bottom color=blue!10]
      (-0.3,0.2) rectangle ++(0.6,-0.2);
    \filldraw[draw=black, top color=blue!50, bottom color=blue!30]
      (-0.5,-0.3) rectangle (0.5,0.3);
  },
}

\forestset{
  declare toks={pic me}{},
  pic dir tree/.style={
    for tree={
      font=\ttfamily,
      grow'=0,
      child anchor=west,
      parent anchor=south,
      anchor=west,
      calign=first,
      inner sep=2pt,
      l=10pt,
      edge path={
        \noexpand\path [draw, \forestoption{edge}]
        (!u.parent anchor) -- +(5pt,0) |- (.child anchor)\forestoption{edge label};
      },
      before typesetting nodes={
        if n=1
          {insert before={[,phantom]}}
          {}
      },
      fit=band,
      tikz+/.code={
        \ifnum\tikzlevel>0
            \pic at (0,0) {folder};
        \fi
        },
    }
  }
}



%% Content starts here.
\begin{document}

\newacronym{AI}{AI}{Artificial Intelligence}
\newacronym{LLM}{LLM}{Large Language Model}
\newacronym{LM}{LM}{Language Model}
\newacronym{IDE}{IDE}{Integrated Development Environment}
\newacronym{NL}{NL}{Natural Language}
\newacronym{NLP}{NLP}{Natural Language Processing}
\newacronym{MBPP}{MBPP}{Mostly Basic Programming Problems}
\newacronym{APPS}{APPS}{Automated Programming Progress Standard}
\newacronym{CLI}{CLI}{Command Line Interface}

\newcommand{\textttbreak}[1]{\texttt{\seqsplit{#1}}}

%---------- Front matter -------------------------------------------------------

\frontmatter

\hypersetup{pageanchor=false} %% Disable page numbering references
%% Render a Dutch outer title page if the main language is English
\IfLanguageName{english}{%
	%% If necessary, information can be changed here
	\degreesought{Professionele Bachelor toegepaste informatica}%
	\begin{otherlanguage}{dutch}%
		\maketitle%
	\end{otherlanguage}%
}{}

%% Generates title page content
\maketitle
\hypersetup{pageanchor=true}

%%=============================================================================
%% Voorwoord
%%=============================================================================
\chapter*{\IfLanguageName{dutch}{Woord vooraf}{Preface}}%
\label{ch:voorwoord}

%% TODO:
%% Het voorwoord is het enige deel van de bachelorproef waar je vanuit je
%% eigen standpunt (``ik-vorm'') mag schrijven. Je kan hier bv. motiveren
%% waarom jij het onderwerp wil bespreken.
%% Vergeet ook niet te bedanken wie je geholpen/gesteund/... heeft
The inspiration for this thesis arose from an internship I completed at StadimData, a company focused on real estate data analytics. While there, I was explicitly instructed not to submit any proprietary code to external \gls{AI} services such as GitHub Copilot or ChatGPT, due to the risk of exposing sensitive business information. This constraint prompted me to explore whether comparable AI‐driven coding support could be achieved entirely on local hardware, without depending on cloud platforms.

Because I needed a thesis topic I could design, implement, and evaluate independently, I settled on an empirical investigation of open‐weight code‐generation models operating on consumer‐grade workstations. This study weaves together system implementation, rigorous benchmarking, and enterprise security requirements into a unified research effort.

I owe particular thanks to Antoine Leflon, whose guidance was invaluable in defining the enterprise constraints and shaping the model‐selection criteria that underpin this work. Without his insight into real‐world deployment needs, I would have struggled to frame a relevant and practical evaluation.
%%=============================================================================
%% Samenvatting
%%=============================================================================

% TODO: De "abstract" of samenvatting is een kernachtige (~ 1 blz. voor een
% thesis) synthese van het document.
%
% Een goede abstract biedt een kernachtig antwoord op volgende vragen:
%
% 1. Waarover gaat de bachelorproef?
% 2. Waarom heb je er over geschreven?
% 3. Hoe heb je het onderzoek uitgevoerd?
% 4. Wat waren de resultaten? Wat blijkt uit je onderzoek?
% 5. Wat betekenen je resultaten? Wat is de relevantie voor het werkveld?
%
% Daarom bestaat een abstract uit volgende componenten:
%
% - inleiding + kaderen thema
% - probleemstelling
% - (centrale) onderzoeksvraag
% - onderzoeksdoelstelling
% - methodologie
% - resultaten (beperk tot de belangrijkste, relevant voor de onderzoeksvraag)
% - conclusies, aanbevelingen, beperkingen
%
% LET OP! Een samenvatting is GEEN voorwoord!

%%---------- Nederlandse samenvatting -----------------------------------------
%
% TODO: Als je je bachelorproef in het Engels schrijft, moet je eerst een
% Nederlandse samenvatting invoegen. Haal daarvoor onderstaande code uit
% commentaar.
% Wie zijn bachelorproef in het Nederlands schrijft, kan dit negeren, de inhoud
% wordt niet in het document ingevoegd.

\IfLanguageName{english}{%
\selectlanguage{dutch}
\chapter*{Samenvatting}

Het voorstel richt zich op het benchmarken van zelf-gehoste \glspl{LLM} om de effectiviteit van deze modellen te evalueren in het veilig ondersteunen van programmeerprocessen binnen bedrijfsomgevingen. Organisaties die gevoelige code verwerken, vermijden steeds vaker cloudgebaseerde \gls{AI}-modellen vanwege het risico dat modellen zoals ChatGPT of Gemini deze code opnemen in hun trainingssets of opslaan, wat leidt tot mogelijke blootstelling van bedrijfsgevoelige informatie.

De onderzoeksvraag luidt:
\begin{quote}
    “Welke zelf-gehoste \gls{LLM} biedt de beste ondersteuning voor codeertaken terwijl datalekrisico’s worden geminimaliseerd?”
\end{quote}

Het doel van de bachelorproef is om een gestructureerde evaluatie te bieden van de prestaties, nauwkeurigheid, efficiëntie en gebruiksvriendelijkheid van geselecteerde open-source \glspl{LLM}, zoals StarCoder en Mistral.

De methodologie omvat het opstellen van gestandaardiseerde benchmarks, zoals HumanEval en \gls{MBPP}, waarmee de modellen worden getest op een reeks programmeertaken in een gecontroleerde on-premise omgeving.

Verwacht wordt dat dit onderzoek niet alleen inzicht geeft in welke modellen technisch het meest geschikt zijn, maar ook praktische aanbevelingen biedt aan bedrijven voor het integreren van veilige \gls{AI}-oplossingen in hun ontwikkelprocessen. Hierdoor worden bedrijven ondersteund in hun zoektocht naar privacybewuste \gls{AI}-oplossingen, waarmee innovatie kan worden bevorderd zonder in te boeten op de veiligheid van gevoelige gegevens.

\selectlanguage{english}
}{}

%%---------- Samenvatting -----------------------------------------------------
% De samenvatting in de hoofdtaal van het document
\chapter*{\IfLanguageName{dutch}{Samenvatting}{Abstract}}

This proposal focuses on benchmarking self-hosted \glspl{LLM} to evaluate their effectiveness in securely supporting programming processes within enterprise environments. Organizations that handle sensitive code are increasingly avoiding cloud-based \gls{AI} models due to the risk that models such as ChatGPT or Gemini may incorporate inputted code into their training sets or store it, leading to potential exposure of sensitive business information.

The research question posed is:  
\begin{quote}
  “Which self-hosted \glspl{LLM} provides the best support for coding tasks while minimizing data-leakage risks?”
\end{quote}

The aim of this bachelor’s thesis is to provide a structured evaluation of the performance, accuracy, efficiency, and usability of selected open-source \glspl{LLM}, such as StarCoder and Mistral.

The methodology involves establishing standardized benchmarks, such as HumanEval and \gls{MBPP}, with which the models are tested on a series of programming tasks in a controlled on-premise environment.

It is expected that this research will not only provide insight into which models are technically most suitable, but also offer practical recommendations to companies for integrating secure \gls{AI} solutions into their development processes. In doing so, companies are supported in their quest for privacy-conscious \gls{AI} solutions, thereby fostering innovation without compromising the security of sensitive data.


%---------- Inhoud, lijst figuren, ... -----------------------------------------

\tableofcontents

% In a list of figures, the complete caption will be included. To prevent this,
% ALWAYS add a short description in the caption!
%
%  \caption[short description]{elaborate description}
%
% If you do, only the short description will be used in the list of figures

\listoffigures

% If you included tables and/or source code listings, uncomment the appropriate
% lines.
\listoftables

\listoflistings

% Als je een lijst van afkortingen of termen wil toevoegen, dan hoort die
% hier thuis. Gebruik bijvoorbeeld de ``glossaries'' package.
% https://www.overleaf.com/learn/latex/Glossaries

%---------- Kern ---------------------------------------------------------------

\mainmatter{}

% De eerste hoofdstukken van een bachelorproef zijn meestal een inleiding op
% het onderwerp, literatuurstudie en verantwoording methodologie.
% Aarzel niet om een meer beschrijvende titel aan deze hoofdstukken te geven of
% om bijvoorbeeld de inleiding en/of stand van zaken over meerdere hoofdstukken
% te verspreiden!

%%=============================================================================
%% Inleiding
%%=============================================================================
\chapter{\IfLanguageName{dutch}{Inleiding}{Introduction}}%
\label{ch:inleiding}

As \gls{AI} rapidly advances, \gls{AI}-powered coding assistants have become increasingly prevalent due to their capacity to significantly enhance productivity, efficiency, and developer workflows. Notably, transformer-based \glspl{LLM}, such as OpenAI's Codex and GitHub Copilot, have demonstrated a remarkable ability to generate code, interpret natural language prompts, and substantially accelerate coding tasks. Developers leveraging these tools report substantial productivity gains, including reduced mental load from repetitive coding tasks and enhanced workflow efficiency \autocite{Microsoft2023Copilot}. Despite these benefits, the widespread adoption of cloud-based \gls{AI} coding assistants by enterprises has been constrained primarily due to data security and privacy concerns.

Recent incidents have highlighted the critical risks associated with cloud-hosted \gls{AI} assistants. Samsung's 2023 decision to prohibit employee use of cloud-based \gls{AI} tools arose after a sensitive internal source code leak occurred via an external \gls{AI} model, highlighting vulnerabilities in how user data is stored and managed externally \autocite{Park2023Samsung}. Similarly, major financial institutions like JPMorgan Chase and Goldman Sachs have proactively restricted the usage of such cloud-based services to mitigate regulatory and confidentiality risks \autocite{Kessel2024}. These incidents underscore the potential for inadvertent data exposure, compliance violations, and loss of intellectual property control associated with external \gls{AI} deployments.

In light of these pressing concerns, one potential solution for enterprises is the deployment of self-hosted \glspl{LLM} for secure coding assistance, running these models internally within their infrastructure. Self-hosted \glspl{LLM} could enhance data privacy and compliance by ensuring operations remain entirely within the organization's environment, eliminating reliance on external services. However, this approach introduces new complexities, including integration challenges, computational resource demands, and ensuring secure, reliable operations within enterprise infrastructure.

\section{\IfLanguageName{dutch}{Probleemstelling}{Problem Statement}}%
\label{sec:probleemstelling}

A central concern driving hesitation among enterprises regarding cloud-based \gls{AI} coding assistants involves the uncertainty surrounding data handling practices. Enterprises worry that their proprietary code or sensitive information could inadvertently become training data, accessible to external entities. Although prominent services like ChatGPT and Google's Gemini provide the ability to disable user data retention, these options require proactive action from users and may not fully mitigate data security risks. Consequently, enterprises face ambiguity regarding their data exposure, reinforcing a cautious stance toward cloud-based \gls{AI} coding tools.

\section{\IfLanguageName{dutch}{Onderzoeksvraag}{Research question}}%
\label{sec:onderzoeksvraag}

\begin{quote}
	\textit{Which self-hosted \glspl{LLM} offer the best support for coding tasks while minimizing data leakage risks?}
\end{quote}
\label{rq:main}

To address this overarching question, we formulated the following \textbf{research sub-questions}:

\begin{enumerate}[label=SQ\arabic*., ref=SQ\arabic*]
	\item \label{sq:secure-dev} Can self-hosted \glspl{LLM} reliably assist in secure software development without introducing vulnerabilities?
	\item \label{sq:performance} How does the performance of self-hosted \glspl{LLM} compare to cloud-based \gls{AI} tools in coding tasks?
	\item \label{sq:deployment} What are the deployment challenges and resource requirements of running \glspl{LLM} locally?
	\item \label{sq:best-practices} What best practices should organizations follow when integrating self-hosted \gls{AI} coding assistants?
\end{enumerate}

\section{\IfLanguageName{dutch}{Onderzoeksdoelstelling}{Research objective}}%
\label{sec:onderzoeksdoelstelling}

The central aim of this research is a comprehensive examination and detailed evaluation of locally-hosted \glspl{LLM} tailored for coding assistance. It rigorously assesses their performance, resilience, and appropriateness relative to cloud-based alternatives. The study focuses explicitly on the capabilities of these self-managed models in executing programming tasks, particularly highlighting use-cases involving critical data confidentiality or proprietary information. Utilizing systematic benchmarking and a comparative methodology, this research seeks to deliver actionable insights into the strengths, constraints, and operational factors associated with adopting local machine learning models. Consequently, the outcomes of this extensive analysis will equip both businesses and individual users with essential guidance, facilitating a clear understanding of whether maintaining in-house machine learning models aligns effectively with their strategic objectives concerning data protection, security considerations, and workflow efficiency.
\section{\IfLanguageName{dutch}{Opzet van deze bachelorproef}{Structure of this bachelor thesis}}%
\label{sec:opzet-bachelorproef}

% Het is gebruikelijk aan het einde van de inleiding een overzicht te
% geven van de opbouw van de rest van de tekst. Deze sectie bevat al een aanzet
% die je kan aanvullen/aanpassen in functie van je eigen tekst.

The remainder of this bachelor's thesis is structured as follows:

Chapter~\ref{ch:stand-van-zaken} provides an overview of the current state of research in the field, based on a literature review.

Chapter~\ref{ch:methodologie} explains the methodology and discusses the research techniques used to address the research questions.

Chapter~\ref{ch:proof-of-concept} justifies the approach of research and provides a comprohensive guide to reproduce the results found in the results Chapter~\ref{ch:results}.

% TODO: Vul hier aan voor je eigen hoofstukken, één of twee zinnen per hoofdstuk

Finally, Chapter~\ref{ch:conclusie} presents the conclusion, answering the overarching research question and sub-questions, and also outlines avenues for future research within this domain.
%%=============================================================================
%% Methodologie
%%=============================================================================
\chapter{\IfLanguageName{dutch}{Methodologie}{Methodology}}%
\label{ch:methodologie}

%% TODO: In dit hoofstuk geef je een korte toelichting over hoe je te werk bent
%% gegaan. Verdeel je onderzoek in grote fasen, en licht in elke fase toe wat
%% de doelstelling was, welke deliverables daar uit gekomen zijn, en welke
%% onderzoeksmethoden je daarbij toegepast hebt. Verantwoord waarom je
%% op deze manier te werk gegaan bent.
%% 
%% Voorbeelden van zulke fasen zijn: literatuurstudie, opstellen van een
%% requirements-analyse, opstellen long-list (bij vergelijkende studie),
%% selectie van geschikte tools (bij vergelijkende studie, "short-list"),
%% opzetten testopstelling/PoC, uitvoeren testen en verzamelen
%% van resultaten, analyse van resultaten, ...
%%
%% !!!!! LET OP !!!!!
%%
%% Het is uitdrukkelijk NIET de bedoeling dat je het grootste deel van de corpus
%% van je bachelorproef in dit hoofstuk verwerkt! Dit hoofdstuk is eerder een
%% kort overzicht van je plan van aanpak.
%%
%% Maak voor elke fase (behalve het literatuuronderzoek) een NIEUW HOOFDSTUK aan
%% en geef het een gepaste titel.

This chapter provides an overview of the methodology applied in this thesis, describing the systematic approach used to answer the research question and achieve the stated objectives. The methodology is structured into three distinct phases, each with clear goals, deliverables, and research techniques, ensuring a robust evaluation of self-hosted \glspl{LLM} for coding assistance.
\section{Research Phases}
\label{sec:research-phases}

\subsection{Phase 1: Literature Study, Model Selection, and Deployment Strategy}

The first phase begins with a structured literature study, presented in Chapter~\ref{ch:stand-van-zaken}, intended to inform the subsequent selection of models, definition of deployment constraints, and identification of suitable benchmarks. This study serves as the theoretical basis for understanding the evolution of \glsentryshort{AI}-based code assistance, considerations around secure software development, and factors relevant to enterprise adoption.

Following the literature study, criteria for model selection and deployment will be established, taking into account technical feasibility, security, and usability within enterprise contexts. Similarly, appropriate benchmarks will be selected to support reproducible and meaningful evaluation across a range of coding tasks.

The planned model selection approach, deployment strategy, and benchmark design are documented in Chapter~\ref{ch:proof-of-concept}, which outlines the framework used to prepare the evaluation setup.

\paragraph{Deliverables of Phase 1:}
\begin{itemize}
	\item A comprehensive literature study on \glsentryshort{AI}-based coding assistance, security considerations, and enterprise adoption.
	\item A curated selection of locally hosted models.
	\item A curated selection of standardized benchmarks for reproducible evaluations.
\end{itemize}

\subsection{Phase 2: Standardized Evaluation of Code Generation Performance}

In the second phase, the selected models will be evaluated through standardized benchmarks designed to test code generation performance. Primary focus will be placed on functional correctness, measured through pass@1 accuracy. Additional attention will be given to prompt stability and generalizability. The solutions generated by the models are automatically validated against predefined test cases to ensure consistency and objectivity.

Evaluations are conducted under uniform resource allocations within a controlled and secure environment, as detailed in Chapter~\ref{ch:results}. Execution of generated code is sandboxed to prevent unintended interactions, ensuring reliability and reproducibility of results.

\paragraph{Deliverables of Phase 2:}

\begin{itemize}
	\item Detailed overview of benchmarking results.
\end{itemize}

\subsection{Phase 3: Enterprise Integration and Comparative Results Analysis}

In the third phase, the benchmarking results will be analyzed to evaluate the practical viability of integrating self-hosted \glspl{LLM} into enterprise software development workflows. This includes assessing how well the models support core development tasks such as code generation and basic editing. A comparative discussion will be undertaken, referencing both proprietary cloud-based systems (e.g., ChatGPT, Gemini) and larger open-weight models that exceed local deployment constraints, to contextualize the capabilities of the tested models. This phase will reflect on trade-offs in performance, scalability, and operational fit from an enterprise perspective, and directly informs the conclusions drawn in Chapter~\ref{ch:conclusie}.

\paragraph{Deliverables of Phase 3:}

\begin{itemize}
	\item Comprehensive comparative analysis of model performance across benchmarking tasks.
	\item Identification and discussion of key challenges and practical limitations associated with self-hosted \glsentryshort{AI} code assistants.
\end{itemize}

\section{Justification of Approach}
\label{sec:justification-of-approach}

The selected research methodology integrates theoretical insights and empirical validation within a structured, phased approach. The literature study establishes foundational knowledge guiding model and benchmark selection. The standardized benchmarking ensures reliable empirical evidence, while the final analysis contextualizes the findings within practical enterprise scenarios.
By combining theoretical foundations with rigorous empirical evaluation, this research provides robust and actionable insights into deploying self-hosted \glspl{LLM} securely and effectively in enterprise coding environments.

\section{Expected Results and Impact}
\label{sec:expected-results-and-impact}

\textbf{Expected Results:} The methodology is expected to yield reproducible benchmark data and insights into the functional reliability and limitations of self-hosted \glspl{LLM} for code generation. It will also help clarify their potential role in real-world development workflows, particularly under constraints relevant to secure enterprise environments.

\paragraph{Impact:}

\begin{itemize}
	\item \textbf{Academic Contribution:} Advances empirical knowledge and methodological clarity regarding \gls{LLM} performance, providing benchmarks and evaluation frameworks for future research.
	\item \textbf{Practical Contribution:} Delivers actionable insights enabling enterprises to make informed decisions regarding the adoption of self-hosted \glspl{LLM}, balancing security, performance, and practical integration considerations.
\end{itemize}
\chapter{\IfLanguageName{dutch}{Literatuurstudie}{Literature Study}}%
\label{ch:stand-van-zaken}

% Tip: Begin elk hoofdstuk met een paragraaf inleiding die beschrijft hoe
% dit hoofdstuk past binnen het geheel van de bachelorproef. Geef in het
% bijzonder aan wat de link is met het vorige en volgende hoofdstuk.

% Pas na deze inleidende paragraaf komt de eerste sectiehoofding.

\section{Historical Evolution of Developer Tools (Pre-AI Era)}
\label{sec:historical-developer-tools}
Before the advent of modern \gls{AI} code assistants, software developers relied on an evolving ecosystem of deterministic tools designed to optimize coding efficiency, correctness, and maintainability. Early \glspl{IDE} introduced features such as syntax highlighting, static type checking, and rule-based autocompletion \autocite{Murphy2006IDEUsage}. Another important category of pre-\gls{AI} developer support tools was static analysis and linting. Static analyzers and linters, including \texttt{lint}, \texttt{Pylint}, and \texttt{ESLint}, further enhanced software quality by identifying style violations, runtime risks, and basic security vulnerabilities without executing code. These utilities accelerated development but operated entirely through manually defined heuristics and static language rules rather than adaptive learning \autocite{Ayewah2008}. The scope of such tools remained entirely diagnostic: they flagged issues but did not attempt to rewrite, transform, or generate code to fix them. This contrast between \emph{non-transformative} tools, which merely flag issues or offer basic, rule-based completions, and \emph{transformative} systems, which can actively generate, rewrite, or adapt code in real time, illustrates the fundamental shift introduced by \glspl{LLM} \autocite{xu2022systematic}. \emph{Non-transformative}  tools are reactive and constrained by predefined heuristics; their suggestions reflect only what developers have manually encoded. \emph{Transformative} \gls{AI} assistants, on the other hand, engage more dynamically with the codebase, extending beyond static rules to contribute meaningfully to the development process.

One of the most dynamic and enduring sources of support has been community-driven platforms like Stack Overflow, which continue to provide curated solutions, though they require manual search and interpretation. Comparative studies show that despite the rise of \gls{AI} chatbots like ChatGPT, many developers still turn to Stack Overflow for trusted, community-vetted answers, especially when depth or discussion is needed \autocite{Liu2023Comparative}.

Some early academic work foreshadowed the shift from static rule-based tools to data-driven developer assistance. Hindle et al.\ demonstrated that source code, like natural language, exhibits strong statistical patterns: developers tend to write predictable sequences of tokens that can be modeled using n-gram language models, which estimate the likelihood of a token based on its preceding context \autocite{Hindle2012}. These findings suggested that code completion could be enhanced using probabilistic methods rather than fixed rules. Subsequent work expanded on this idea, with researchers like Allamanis et al.\ exploring statistical models for tasks such as API suggestion and bug detection \autocite{Allamanis2018}. Another strand of research used similar techniques to generate natural-language comments from source code, further highlighting the parallel between code and language modeling \autocite{MovshovitzAttias2013}.

However, these early approaches were limited to surface-level patterns. They lacked semantic understanding, operated at the token level, and were incapable of generating coherent code from scratch. Before the rise of deep learning, learned models could only augment traditional tools in narrow ways, offering token predictions or comment suggestions, but not synthesizing new logic or adapting to a developer’s broader intent.

\section{Emergence of AI-Powered Code Assistance}
\label{sec:emergence-ai-code-assistance}

Building upon the earlier use of deterministic and statistical developer tools described in Section~\ref{sec:historical-developer-tools}, the late 2010s marked a decisive shift toward learned representations and deep learning. This transition was catalyzed by the introduction of Transformer architectures, which utilize self-attention mechanisms to capture long-range dependencies within sequences. Originally developed for \gls{NLP} tasks, Transformers have proven effective in modeling the syntactic and semantic structures of source code, which, like natural language, exhibits sequential and hierarchical patterns. Researchers began adapting these architectures to code-related tasks, such as code completion, function naming, and bug fixing, demonstrating that Transformers could learn meaningful representations of code syntax and semantics at scale . This advancement enabled models to surpass the limitations of traditional n-gram models and rule-based systems, facilitating more sophisticated and context-aware code generation and analysis \autocite{Chirkova2021}.

This development gave rise to transformer-based models such as CodeBERT~\autocite{Feng2020} and CodeT5~\autocite{wang2021codet5}, which were pretrained on large corpora of code and natural language, and then fine-tuned for downstream tasks like summarization, defect detection, and code generation. These models demonstrated that transformer architectures could be successfully adapted to the structural complexities of code, bridging natural language understanding and code synthesis within a shared representational framework.

By 2019, prototype autoregressive models for real-time code completion began entering developer environments. TabNine, for instance, incorporated a GPT-2 model into an \gls{IDE} plugin, demonstrating that transformer-based \glspl{LLM} trained on code could generate fluent, context-aware suggestions in real-time coding workflows \autocite{TabNineBlog2019}. This marked one of the first practical demonstrations that large pretrained models could assist developers inside the editor, not just in isolated research settings.

The field saw a major leap forward with the release of OpenAI’s Codex in 2021, a Transformer-based \gls{LLM} fine-tuned on billions of lines of public source code \autocite{Chen2021}. Codex could translate natural-language instructions into executable code and complete sizable functions from brief descriptions. Its performance was evaluated on the HumanEval benchmark, a set of hand-written Python programming problems with associated unit tests designed to assess functional correctness. On this benchmark, Codex achieved 28.8\% pass@1 and 70.2\% pass@100, indicating that it could generate correct solutions on the first attempt in nearly one-third of cases, and given 100 attempts, solve over 70\% of tasks \autocite{Chen2021,Austin2021}. These scores far surpassed those of previous models and cemented Codex as a state-of-the-art tool for code synthesis.

That same year, GitHub Copilot~\autocite{Github2021CopilotAnnouncement} was launched as a commercial integration of Codex, embedding the model into popular \glspl{IDE} like VSCode. This marked a watershed moment: \gls{AI} assistants had graduated from prototype to production, entering the daily workflows of thousands of developers. Copilot could suggest entire lines or blocks of code, adapting to the context of the file and the developer's comments, essentially acting as a real-time ``\gls{AI} pair programmer.'' This breakthrough catalyzed efforts across industry and academia. DeepMind released AlphaCode, a system that tackled competitive programming problems by generating and ranking thousands of code candidates, achieving a top-54\% rank among human competitors \autocite{Li2022}. Google developed models like PaLM-Coder and Codey, while Amazon introduced CodeWhisperer, a commercial code assistant integrated with AWS developer tooling \autocite{codewhisperer2023}.

The open-source community also responded. Meta’s research division released Code Llama, a family of Transformer models fine-tuned for code generation and trained on massive code corpora, built upon their earlier LLaMA models \autocite{codellama2023}. Similarly, BigCode’s StarCoder initiative produced competitive open-weight models, democratizing access to high-quality code generation tools.

The early 2020s thus marked a turning point: \gls{AI}-powered developer tools evolved from academic experiments to integrated assistants, reshaping workflows and accelerating a shift toward natural-language-driven programming. Transformer-based models moved from being theoretical explorations to practical infrastructure embedded within the software development lifecycle.

\section{Productivity Gains with AI-Enhanced Developer Workflows}
\label{sec:productivity-gains}
The introduction of \gls{AI} coding assistants has led to measurable improvements in developer productivity, as evidenced by both controlled studies and industry surveys \autocite{Ziegler2025}. By automating boilerplate coding and offering relevant suggestions, these tools allow developers to focus on higher-level logic and design.

A notable controlled experiment by Peng et al. (2023) provided empirical evidence of substantial productivity gains: developers were tasked with building an HTTP server, and those with access to GitHub Copilot completed the task 55.8\% faster than those without \gls{AI} assistance \autocite{Peng2023}. This study, involving a treatment group using Copilot and a control group coding manually, demonstrated that \gls{AI} suggestions can significantly reduce development time for certain tasks. The assisted developers not only finished faster on average, but many also reported experiencing less frustration on routine coding steps, since the \gls{AI} handled repetitive code and syntax details.

Beyond speed, \gls{AI} assistants appear to positively impact software quality and developer satisfaction in some contexts. Microsoft's research indicates that a majority of developers feel more productive and are willing to rely on \gls{AI} for routine tasks. For instance, a large-scale survey reported that 70\% of Copilot users felt more productive, and 77\% preferred not to work without such an \gls{AI} assistant once accustomed to it \autocite{Microsoft2023Copilot}.

The integration of \gls{AI} coding assistants has introduced a paradigm shift in software development workflows. By enabling developers to articulate coding intentions through natural-language prompts, these tools facilitate the rapid generation of corresponding code snippets, effectively allowing developers to operate at a higher level of abstraction during the initial stages of development. This capability accelerates prototyping and solution exploration, as developers can quickly iterate on ideas without delving into low-level implementation details. Moreover, by automating repetitive tasks and providing contextually relevant code suggestions, \gls{AI} assistants allow for developers to concentrate on design and problem-solving aspects of their projects. This evolution from manual coding to intent-driven development has been a significant factor in the widespread adoption of \gls{AI} tools across the software industry.

\section{Limitations and Risks of AI Code Assistants}
\label{sec:ai-limitations}
Despite their impressive capabilities, current \gls{AI} code assistants come with notable limitations and risks. Productivity gains are not universal across all users or coding contexts. The same study by Peng et al. (2023) discussed in~\ref{sec:productivity-gains}, also reported on mixed outcomes alongside its findings on productivity gains: developers with more experience or those adept at prompt engineering benefitted the most, while some novices struggled to evaluate \gls{AI}-generated code for correctness. This indicates that effective use of \gls{AI} tools requires a new set of skills, such as writing good descriptive prompts and critically reviewing outputs, rather than blindly accepting suggestions \autocite{Peng2023}.

Moreover, traditional productivity metrics like lines of code written or task completion speed may not fully capture the nuanced changes in developer workflows introduced by \gls{AI} assistants. As Haque et al. (2025) point out, these conventional measures often overlook the cognitive shifts and new forms of collaboration that arise when developers interact with \gls{AI} tools \autocite{Haque2025}.

Meanwhile, the primary technical concern centers on the accuracy and correctness of code generated by \glspl{LLM}. These models can and often do produce syntactically valid yet semantically incorrect or suboptimal code, particularly when faced with ambiguous prompts or inputs that lie outside their training distribution. Unlike a human collaborator, an \gls{AI} lacks genuine understanding of a developer’s intent and instead generates code based on learned statistical patterns. As a result, it may introduce subtle bugs or flawed logic that elude initial review. Dou et al. (2024) conducted an extensive evaluation of \gls{LLM}-generated code and found that functional bugs, code that runs but fails to fulfill its intended task, were the most common error type \autocite{Dou2024}. This highlights the risk of accepting \gls{AI}-generated suggestions without rigorous testing or human oversight, as such code may appear correct on the surface while introducing latent defects into the software.

Furthermore, \gls{AI} models have a tendency to ``hallucinate''---i.e., produce outputs that are contextually plausible but factually or logically incorrect. In coding, a hallucination might manifest as a call to a non-existent API or the use of an incorrect algorithm \autocite{Tian2024}. Such mistakes require vigilant oversight by the developer, partially offsetting the productivity gains.

Security and compliance risks are another critical issue. Studies have shown that code generation models may inadvertently produce insecure code patterns. Pearce et al. evaluated Copilot’s outputs on security-sensitive programming tasks and found that a significant fraction of the suggestions contained vulnerabilities such as SQL injection, buffer overflows, or use of outdated cryptography \autocite{Pearce2022}. The \gls{AI}, having been trained on a vast corpus of public code, which itself includes insecure code, does not inherently distinguish between safe and unsafe practices. It may regurgitate common but dangerous coding patterns. This raises concerns for organizations about introducing vulnerabilities through \gls{AI}-recommended code. Moreover, since these models lack an understanding of secure coding guidelines, they will not automatically sanitize inputs or enforce security checks unless explicitly prompted to do so.

Legal and ethical risks have also been identified. \gls{AI} assistants trained on open-source code have been observed to sometimes reproduce snippets of that code verbatim in their suggestions. If those snippets are under restrictive licenses (e.g., GPL), using them without attribution could violate license terms. This issue of ``code memorization'' was highlighted soon after Copilot’s launch, leading to debate on whether \gls{AI}-generated code might inadvertently carry over copyright from its training data \autocite{Sandoval2022}. A related ethical concern is attribution, developers using \gls{AI} suggestions might not realize when a large suggested block actually comes from a specific source, potentially failing to give credit where due \autocite{Xu2024}. These uncertainties have even led to ongoing litigation challenging the legality of training on open code and the outputs produced, highlighting unresolved questions in this space.

Beyond correctness, security, and legal compliance, there are broader impacts on developer behavior to consider. Over-reliance on \gls{AI} suggestions could lead to erosion of fundamental programming skills or understanding, as developers might accept answers without fully reasoning through the problem. There is also the risk of cognitive bias: if an \gls{AI} often suggests a particular solution approach, it might narrow a developer’s exposure to alternative, potentially more innovative solutions. Recent work has also shown that developers perceive \gls{AI} assistants as confident and helpful, but sometimes blindly trust them more than traditional community-based platforms, raising concerns about misplaced confidence and reduced scrutiny of suggestions \autocite{Li2023QA}.

Additionally, initial reports indicate that indiscriminate use of \gls{AI}-generated code can increase ``code churn'' ,  developers may spend additional time later rewriting or cleaning up \gls{AI}-produced code that was not optimal in the first place, especially if it leads to integration problems with the rest of the codebase. In one study, teams using \gls{AI} assistance saw a higher rate of edits and refactors post-code-review, suggesting that some of the time saved upfront may be lost in downstream corrections \autocite{GitClear2024Copilot}.

While these tools are powerful, they remain just that, tools. A competent \gls{AI} coding assistant will not take the place of the developer, and understanding their limitations is crucial. Responsible use entails treating the \gls{AI} as a junior developer: helpful and fast, but in need of supervision, testing, and guidance to avoid the aforementioned risks.

\section{Motivations for On-Premises AI Deployment: Privacy and Compliance}
\label{sec:on-prem-motivations}
The rise of cloud-based \gls{AI} coding assistants has triggered serious privacy and confidentiality concerns, especially within enterprises handling proprietary code. Services like GitHub Copilot and Amazon CodeWhisperer rely on sending the developer’s prompt (which may include source code and descriptions of tasks) to remote servers where the \gls{AI} processing occurs \autocite{GithubCopilotArchitecture,AWSCodeWhispererPromptEngineering}. For companies, this architecture poses a risk: sensitive source code or trade secrets might be exposed to a third-party and potentially retained in logs or used to further train the \gls{AI}. Even if service providers pledge not to misuse client data, the mere transmission of code outside the corporate firewall alone is already seen as a liability in many industries.

High-profile incidents have underscored these fears. In April 2023, it was reported that engineers at Samsung accidentally leaked confidential semiconductor source code by pasting it into ChatGPT, after which the company discovered the data could no longer be retracted from OpenAI’s servers \autocite{Park2023Samsung}. This incident led Samsung to swiftly ban the use of external \gls{AI} tools like ChatGPT for any internal code or data, citing the inability to control or delete information once it leaves their network.

Samsung’s reaction is not isolated. A wave of organizations, particularly in finance and government, have instituted preemptive restrictions on cloud \gls{AI} usage. Major banks such as JPMorgan Chase, Bank of America, and Citigroup in early 2023 either forbade or heavily limited employees from inputting company data into ChatGPT or similar tools \autocite{Kessel2024}. Their concern centers on regulatory compliance and client confidentiality, any leak of personally identifiable information or strategic code could be devastating and potentially breach regulations like the GDPR or industry-specific data protection rules.

Furthermore, many jurisdictions and regulatory bodies have started scrutinizing how \gls{AI} services handle user-provided data. For example, Italy’s data protection authority briefly banned ChatGPT in 2023 until assurances about data handling were given, signaling that even government regulators view the unfettered cloud \gls{AI} data flow as problematic \autocite{Reuters2023}. These privacy concerns have fueled a broader reconsideration of \gls{AI} deployment strategies, with organizations increasingly moving away from cloud reliance toward internal, locally controlled solutions for handling sensitive data. Companies are increasingly asking whether the convenience of a hosted \gls{AI} service is worth the potential exposure of their intellectual property. As a result, demand has grown for \gls{AI} solutions that give organizations greater control. This demand is evident in new offerings by major \gls{AI} providers themselves: for instance, Microsoft announced plans for an enterprise-grade version of its Azure OpenAI Service, offering private deployment of language models like GPT-4 on dedicated Azure infrastructure. This ensures customer data remains isolated and is not used for model training \autocite{MicrosoftAzureOpenAIPrivateDeployment}. Such moves illustrate that even cloud \gls{AI} vendors acknowledge the need for data privacy safeguards to convince enterprise customers. Growing concerns over privacy and confidentiality have become a pivotal factor in prompting organizations to rethink their reliance on cloud-based \gls{AI} tools, accelerating interest in on-premises and more controllable alternatives.

\subsection{Case Studies: Transitions from Cloud to On-Premises AI Solutions}
\label{sec:case-studies}
To illustrate this trend, we highlight several concrete examples of organizations shifting from cloud-based \gls{AI} code assistants to on-premises or otherwise controlled solutions are beginning to emerge as the technology matures and concerns mount.

\begin{itemize}
	\item \textbf{Case 1:} \textbf{Samsung:} As stated above, after incidents where sensitive code was exposed via ChatGPT, Samsung not only banned external \gls{AI} platforms but reportedly began exploring the development of an internal \gls{AI} system for coding assistance, to be hosted on its own servers. This internal system would allow Samsung developers to enjoy some benefits of \gls{AI} suggestions without any code ever leaving the corporate network. While details of Samsung’s in-house solution remain private, the case exemplifies how a large enterprise pivoted in response to privacy risks: from tentative adoption of a public \gls{AI} tool to investing in a proprietary, locally hosted \gls{AI} assistant aligned with its compliance needs \autocite{sharma2023samsung}.
	\item \textbf{Case 2:} \textbf{JPMorgan Chase:} In February 2023, JPMorgan became one of the first financial institutions to restrict employees from using ChatGPT, due to concerns about sensitive financial code and data being shared externally. Instead of abandoning \gls{AI} assistance altogether, JPMorgan took a different approach: it built an internal \gls{AI} assistant named ``LLM Suite.'' By 2024, JPMorgan’s \gls{LLM} Suite was rolled out to 60,000 employees as a behind-the-firewall solution for generative \gls{AI} tasks. \gls{LLM} Suite acts as a portal to large language models in a way that, as JPMorgan’s Chief Data and Analytics Officer Teresa Heitsenrether noted, ``we can leverage the model while still keeping our data protected''~\autocite{Kessel2024}. In practice, this means the \gls{AI} (which is based on models from OpenAI but hosted in a dedicated environment) can be used to summarize documents or suggest code, yet all prompts and responses remain confined to JPMorgan’s infrastructure or a trusted cloud tenant that does not commingle data with others. This case demonstrates a transition from a general-purpose cloud \gls{AI} to a tailored on-premises (or private-cloud) \gls{AI} solution in order to meet strict compliance standards.
	\item \textbf{Case 3:} \textbf{Los Alamos National Laboratory (LANL):} In November 2023, LANL expanded its partnership with SambaNova Systems to enhance its generative \gls{AI} and large language model \gls{LLM} capabilities. This expansion included scaling up the deployment of SambaNova's DataScale systems and adopting the SambaNova Suite, a comprehensive, enterprise-optimized \gls{AI} platform designed for on-premises deployment. By hosting these solutions locally, LANL can ensure stringent data security, maintain full operational control, and comply with rigorous regulatory standards. This decision illustrates a broader trend among security-conscious organizations toward self-hosted deployments leveraging open-source models to protect data integrity and privacy \autocite{SambaNova2023LANL}.
\end{itemize}

Beyond individual companies, there is a growing shift toward on-premises \gls{AI} deployments in sectors with elevated privacy, security, or regulatory demands. Organizations in areas such as healthcare, finance, and national security are increasingly cautious about exposing sensitive data to third-party cloud services, prompting interest in self-hosted \glspl{LLM} and locally managed \gls{AI} infrastructure. In the public sector, for instance, several European governments in France and Germany have emphasized the importance of ``sovereign AI'' ,  systems that can be deployed in national data centers to ensure data autonomy and compliance with regional privacy laws. A notable example is the European GAIA-X initiative, which aims to build a federated and secure data ecosystem for cloud and \gls{AI} services within Europe’s jurisdiction. Among its goals is to promote digital sovereignty by supporting infrastructures that can be hosted entirely within a country’s own boundaries \autocite{GaiaXHub2024}. While not specific to code generation, these efforts reflect a broader movement: driven by privacy concerns and regulatory obligations, many organizations are transitioning from general-purpose, cloud-based \gls{AI} to more controlled, locally managed deployments. Though such transitions can involve performance and maintenance trade-offs, the imperative for data control remains a dominant force shaping enterprise adoption strategies.

\section{Local AI Adoption for Privacy and Compliance}
\label{sec:local-ai-for-privacy}
The push for locally hosted \gls{AI} code assistants stems from the desire to reconcile the benefits of \gls{AI} with stringent privacy and compliance requirements. In an enterprise setting, a ``locally hosted'' \gls{AI} typically means the model is run either on individual developers’ workstations, hosted on  company’s on-premises servers, or in a private cloud instance under the company’s control. This ensures that source code and prompts used for \gls{AI} never leave the trusted environment.

Over the last two years, the feasibility of such local \gls{AI} deployments has significantly improved. One catalyst was the release of powerful open-source \glspl{LLM}. For instance, Meta’s LLaMA and LLaMA 2 models \autocite{Touvron2023} demonstrated that smaller-scale (7B--70B parameter) models, when trained on large corpora including code, could achieve performance on par with some proprietary models. LLaMA’s release under a permissive license in 2023 (especially LLaMA 2 which is available for commercial use) enabled organizations to take these base models and fine-tune them on their own code repositories securely.

Following Meta’s move, other open initiatives like BigCode’s StarCoder provided highly capable code-specific models openly \autocite{Li2023StarCoder}. StarCoder, a 15-billion-parameter model trained on public GitHub code, has a performance approaching that of Codex on many tasks, and being open-source, it can be run locally or even modified to better suit an enterprise’s coding style and libraries.

Adopting a locally hosted \gls{AI} code assistant is still an experimental endeavor for most enterprises, as it comes with challenges in setup, maintenance, and ensuring quality. Unlike a managed cloud service, running an \gls{LLM} in-house requires significant computational resources (GPUs or specialized \gls{AI} hardware) and expertise in machine learning engineering. Companies venturing into this space often start with pilot projects. For example, an enterprise might set up an instance of an open-source model (like Code Llama or StarCoder) on a high-end server and integrate it with an internal \gls{IDE} plugin for developers to test. Initial trials help in assessing whether the model’s suggestions are accurate enough and whether latency is acceptable for real-time usage.

In many cases these locally hosted models need further fine-tuning or prompt engineering to specialize them for the company’s codebase or coding standards. This fine-tuning process can be conducted entirely within the organization’s own infrastructure, ensuring that proprietary code remains internal while enabling the development of a model specifically tailored to its coding practices, thereby creating their own custom model. Recent studies have shown that fine-tuning a code \gls{LLM} on a domain-specific corpus (for example, a particular company’s repositories) can improve its relevance and reduce hallucinations in that domain, though care must be taken to avoid overfitting \autocite{Roziere2023}.

From a compliance perspective, locally hosted \gls{AI} offers clear advantages. Since the entire inference process stays within the organization, there is a much lower risk of a data breach or unintended data retention by third parties. Companies in sectors like finance and healthcare, bound by regulations such as HIPAA or PCI-DSS, find this control necessary to even consider using \gls{AI} for code assistance. Additionally, hosting models locally enables auditability: the organization can log every prompt and completion, control versioning of the model, and ensure that outputs can be traced and reviewed, which is helpful for explaining decisions or conducting code audits.

Some organizations also see strategic benefits in building internal \gls{AI} capabilities, it reduces dependency on external vendors and can be tailored to internal development practices (for example, an on-prem assistant could be instructed to always enforce certain secure coding guidelines). Nevertheless, it remains experimental in the sense that the ecosystem for on-prem code \glspl{LLM} is still maturing. The rapid pace of \gls{AI} research means that open models are improving quickly, but keeping up-to-date with the latest techniques (such as model quantization for efficiency, or applying reinforcement learning from human feedback for better alignment) is non-trivial for a company whose core business is not \gls{AI}. Despite these challenges, all these developments point to locally hosted \gls{AI} code assistants becoming a viable option for enterprises that require privacy and compliance, effectively bridging the gap between advanced \gls{AI} capabilities and the strict governance policies those enterprises must uphold.

\section{Cloud vs. Local AI: Landscape and Trends}
\label{sec:ai-trends}
The competitive landscape for \gls{AI} coding assistants now spans a spectrum from fully cloud-based services to fully local solutions, with hybrid approaches in between. On one end, we have cloud-hosted proprietary models like OpenAI’s Codex (via GitHub Copilot) and Google’s Codey, which benefit from massive scale and continual improvements by specialized \gls{AI} teams. On the other end, we have open-source or self-hosted models that organizations can run independently.

A key trend in recent years is the narrowing of the performance gap between open models and proprietary ones, especially for code tasks. Research contributions such as StarCoder and Code Llama have shown that with sufficient data and tuning, open models can achieve competitive pass@1 scores on standard benchmarks \autocite{Li2023StarCoder}. For example, Code Llama (34B parameters) was reported to reach near state-of-the-art performance on the HumanEval benchmark, approaching the capabilities of OpenAI’s early Codex models.

Likewise, community-driven fine-tunes such as WizardCoder~\autocite{Luo2023} have demonstrated competitive performance. Trained through instruction tuning, where models are optimized to follow natural language prompts and generate task-relevant outputs, WizardCoder, despite being openly available, has in some programming tasks slightly outperformed proprietary models. Notably, some of these gains were achieved by incorporating filtered outputs from those same closed models. This narrowing of the performance gap has prompted more organizations to view local \gls{AI} solutions as viable alternatives rather than inferior fallbacks.

In terms of market trends, cloud \gls{AI} assistants currently enjoy widespread adoption due to ease of use. GitHub reported that as of 2023, over 50,000 organizations (including one-third of Fortune 500 companies) had adopted Copilot in some capacity \autocite{Opsera2024}. These services often integrate seamlessly with popular development environments and offer features like cloud-based model updates, team management, and telemetry that make them attractive to enterprise IT departments from a convenience standpoint.

However, the very ubiquity of cloud solutions has driven differentiation. Companies like Amazon and Google have introduced their own \gls{AI} coding assistants, Amazon Q Developer and Gemini Code Assist Enterprise respectively, not only to compete on quality but also to address enterprise data governance preferences. Amazon Q Developer, which incorporates the functionality of the earlier CodeWhisperer tool, includes a Professional tier explicitly designed for organizational use: it does not retain or use user code for model training, positioning it as a more enterprise-friendly alternative to Copilot, particularly for companies with strict data privacy requirements~\autocite{AmazonQDeveloper}. Similarly, Google's Gemini Code Assist Enterprise offers deep local codebase awareness supported by a large token context window, enabling more relevant code generation and transformation tailored to enterprise applications~\autocite{google2024gemini}.

At the same time, Microsoft and OpenAI, facing pressure from privacy-conscious clients, have begun offering private deployments (e.g., Azure OpenAI Service allowing a dedicated instance of Codex/ ChatGPT for a client) \autocite{MicrosoftAzureOpenAIPrivateDeployment}. The pricing models also differ: cloud assistants are typically subscription-based or pay-as-you-go, whereas a local solution has upfront hardware costs but potentially lower marginal cost per use if scaled. Moreover, a small development team or for individual developers locally hosting open models are a much more realistic endeavor than paying for such enterprise solutions solely for privacy concerns.

Another important aspect of the cloud vs local comparison is iteration speed and customization. Cloud providers can roll out model improvements or new features (like vulnerability filters or multiline completions) to all users at once. Local models, in contrast, give the user the freedom to customize ,  for example, a company could prioritize a local model to prefer certain coding styles or internal libraries by fine-tuning it, a level of customization that a one-size-fits-all cloud service cannot provide.

A growing trend involves hybrid deployment strategies, where organizations employ cloud-based \gls{AI} assistants for general development tasks while reserving local models for sensitive or proprietary code. This ``dual'' setup allows enterprises to balance convenience with confidentiality. Development environments are increasingly incorporating support for such configurations, enabling automatic routing of inference requests to either local or cloud endpoints based on data classification or policy constraints \autocite{PALO}.

Looking ahead, the competitive landscape will likely be shaped by upcoming regulatory changes, potentially favoring local solutions if stricter data locality laws emerge, and by technological advances that improve model efficiency, possibly making it feasible to run powerful models on a single server or even a developer’s laptop, greatly boosting the appeal of locally hosted \glspl{LLM}. In parallel, the open-source community continues to drive advancements in benchmarking and transparency. Openly released models are frequently accompanied by detailed evaluation metrics and benchmark comparisons, which can illuminate areas where proprietary systems underperform. This transparency has become a catalyst for broader improvements across both open and closed model ecosystems.

The current landscape is highly dynamic: while cloud \gls{AI} solutions still dominate in adoption and raw performance, local alternatives are advancing quickly, fueled by open innovation and growing demands for data control. Organizations now face a context-dependent decision, weighing trade-offs between compliance, cost, customization, and access to cutting-edge capabilities.

\section{Benchmarking LLMs for Code Generation and Developer Assistance}
\label{sec:llm-benchmarks}
\glspl{LLM} are typically evaluated on code generation through measures of functional correctness, where a model's outputs are validated against test cases. The predominant metric in academic literature is pass@k, the probability that at least one of $k$ generated outputs passes all test cases. This approach, first explored by Kulal et al (2019).~\autocite{kulal2019spoc} and formalized in the context of \glspl{LLM} by Chen et al (2021).~\autocite{Chen2021}, has become the standard for evaluating code synthesis models. Researchers emphasize that unlike surface-level metrics such as BLEU or ROUGE, execution-based validation directly reflects a model's utility for software development tasks~\autocite{xu2022systematic}. Consequently, most benchmarks present a natural language prompt and use predefined test suites to assess correctness, with little focus on non-functional metrics such as code style or runtime efficiency.

This section surveys benchmarks designed to evaluate \glspl{LLM} on code generation and developer assistance. Particular attention is paid to those benchmarks that are publicly available, rely on execution-based evaluation, and are most applicable to secure or locally deployed systems. Benchmarks focused solely on general-purpose natural language tasks are omitted unless their structure directly supports coding-related evaluation.

\subsection{HumanEval}
HumanEval was introduced as part of OpenAI's evaluation suite for Codex~\autocite{Chen2021}. It comprises 164 hand-written Python problems, each consisting of a function signature, a natural language specification in the form of a docstring, and an associated set of unit tests. HumanEval tasks typically require implementing a single function, with problems drawn from domains like algorithm design, string processing, and numerical computation. By design, the dataset avoids overlap with publicly available code repositories, reducing the likelihood of data leakage into pretraining corpora. The benchmark’s clarity and compact scope make it particularly suitable for assessing first-attempt correctness. Pass@1 is often cited as the most informative metric, reflecting the reliability of a model in producing correct solutions without needing to sample multiple times. Codex, for example, achieved 28.8\% pass@1 and 70.2\% pass@100 on HumanEval, demonstrating how much performance can improve when multiple completions are allowed. Because of its concise structure, HumanEval is especially practical for evaluating models deployed in local environments, where sampling may be limited due to resource constraints. Nevertheless, its Python-only scope and relatively small size have spurred the development of complementary benchmarks.

\subsection{MBPP}
The \Gls{MBPP} benchmark was created to address some of the limitations inherent in HumanEval~\autocite{Austin2021}. Comprising 974 crowd-sourced Python problems aimed at entry-level programmers, \gls{MBPP} offers greater diversity and scale. Unlike HumanEval’s docstring-style prompts, \gls{MBPP} presents natural-language problem descriptions that more closely resemble the way users might phrase queries to a code assistant. This makes \gls{MBPP} particularly relevant for evaluating models in interactive, assistance-focused contexts. Both benchmarks share a similar evaluation procedure: code outputs are validated using hidden unit tests, and performance is reported in terms of pass@k. However, \gls{MBPP} introduces more variation in linguistic formulation and includes a broader set of programming constructs, such as loops, data structures, and string manipulations. While HumanEval remains more controlled, \gls{MBPP}’s variability provides a richer testbed for assessing how well models generalize across tasks. That said, \gls{MBPP} also inherits certain limitations, most notably, its restriction to Python and the possibility that some problem solutions might appear in web-scraped training datasets.

\subsection{APPS}
Whereas HumanEval and \gls{MBPP} focus on isolated function synthesis tasks, the \Gls{APPS} benchmark expands the problem space to full program generation~\autocite{Hendrycks2021}. With 10,000 Python problems collected from competitive programming platforms such as Codeforces and LeetCode, \gls{APPS} challenges models with tasks that require parsing complex problem descriptions and assembling multi-step solutions. The evaluation remains grounded in execution-based metrics, but the shift from single-function synthesis to script-level generation introduces new difficulties in reasoning, control flow, and input/ output formatting. Notably, \gls{APPS} is organized by difficulty level, allowing researchers to examine model performance across a spectrum from beginner to expert. While simpler problems resemble those in \gls{MBPP}, the harder categories introduce dynamic programming, greedy algorithms, and graph traversal, which demand algorithmic planning and domain-specific knowledge. Even strong models, such as GPT-Neo 2.7B, have shown modest pass@1 rates of around 20\% on easier subsets, highlighting the benchmark’s difficulty. Despite its strengths, \gls{APPS} poses certain challenges for developer assistance evaluation. Generating a standalone solution script differs substantially from providing inline suggestions or contextual completions within an IDE. Thus, while \gls{APPS} reveals upper limits of generative capability, it may not reflect the day-to-day interaction patterns expected from a local coding assistant.

\subsection{CodeContests (AlphaCode)}
DeepMind’s CodeContests dataset, built to evaluate the AlphaCode system, is similar in spirit to \gls{APPS} but focuses specifically on competition-grade programming tasks \autocite{Li2022}. The dataset includes problems from contests like Codeforces and AtCoder, with evaluation centered on pass@k within a realistic submission budget, typically, 10 samples per problem. AlphaCode’s model, for example, achieved a 34\% success rate under these constraints, a notable result given the complexity of the tasks. CodeContests employs a rigorous temporal split and multiple test cases per problem, reducing risks of contamination and inflating scores. While its competitive focus makes it an excellent stress test for generalization and problem-solving, it shares \gls{APPS}’s limitations for evaluating interactive code assistance. The tasks emphasize standalone solution writing rather than incremental collaboration or partial completions. For secure, on-premise models focused on developer support, CodeContests serves as a valuable upper-bound test but may be less reflective of everyday development workflows.

\subsection{MultiPL-E}
The MultiPL-E benchmark extends the evaluation landscape by translating existing Python benchmarks like HumanEval and \gls{MBPP} into 18 programming languages~\autocite{Cassano2022}. This allows researchers to assess the cross-lingual generalization capabilities of \glspl{LLM}. Each problem maintains its original intent and is paired with language-specific test cases for evaluation. Languages span multiple paradigms, including object-oriented (Java, C\#), functional (Haskell), and scripting (JavaScript), and thus expose the syntactic and idiomatic robustness of code models. MultiPL-E's structure enables direct comparison across languages using parallel tasks. In practice, models often perform best in Python and JavaScript, languages overrepresented in training corpora, while performance may degrade in underrepresented languages. For local code assistants operating in enterprise settings, where legacy or less common languages are in use, MultiPL-E provides essential insights into model readiness. Its reliance on automatically translated tasks introduces some risk of semantic drift, but its breadth makes it a compelling complement to Python-centric evaluations.

\subsection{DS-1000}
DS-1000 was introduced to assess a distinct domain: data science and scientific computing~\autocite{Lai2022DS1000}. Sourced from StackOverflow, its 1,000 tasks target the use of libraries like NumPy, Pandas, Matplotlib, and scikit-learn. Evaluation combines execution with structural constraints, ensuring that solutions not only produce correct outputs but also use appropriate library functions and avoid hardcoded values. This multi-criteria protocol increases evaluation fidelity: Lai et al. report that only 1.8\% of accepted solutions were false positives. The problems are grounded in realistic data manipulation and analysis tasks, making the benchmark highly applicable to industry environments where such workflows are common. Unlike HumanEval or \gls{MBPP}, DS-1000 tests not just algorithmic thinking but also familiarity with specialized APIs, offering a more practical measure of a model’s readiness for data-oriented developer support. However, it remains Python-exclusive and domain-specific, serving best as a complement to more general-purpose benchmarks.

\subsection{Extended and Related Benchmarks}
Recent efforts have sought to augment existing benchmarks or explore new problem types. EvalPlus, introduced by Liu et al.~\autocite{Liu2023Rigorous}, enhances functional evaluation rigor by expanding the test suites of widely used benchmarks such as HumanEval and \gls{MBPP}. These augmented variants, HumanEval+ and MBPP+, retain the original prompts and function signatures but attach significantly more comprehensive unit tests per task. For example, HumanEval+ extends each of the 164 original problems to include over 80 test cases, synthesized through a combination of LLM-based generation and mutation strategies. \gls{MBPP}+ applies the same methodology to a filtered subset of 399 \gls{MBPP} problems with clearly defined functionality.

This augmentation addresses a key shortcoming of traditional benchmarks: sparse test coverage. With only a few test cases, models can appear accurate despite failing to generalize beyond simple inputs. EvalPlus mitigates this by enforcing correctness over a much broader input space, leading to a more reliable pass@1 metric. In doing so, it also enables the assessment of prompt stability, i.e., whether a model continues to produce valid outputs when faced with paraphrased or structurally varied inputs.

Initial studies report performance declines exceeding 10 percentage points when switching from base to augmented benchmarks, particularly for models optimized on narrow test suites~\autocite{Liu2023Rigorous}. This reveals a brittleness not captured by conventional evaluations and underscores the importance of more exhaustive functional testing, especially in high-stakes applications where first-attempt correctness is critical.

HumanEval+ and \gls{MBPP}+ now serve as valuable complements to their base versions, providing greater diagnostic depth without sacrificing comparability. Their design makes them especially relevant for secure and resource-constrained environments, where robust behavior and test-time generalization are essential.

Other recent benchmarks explore distinct challenges. HumanEvalPro and MBPP-Pro, introduced by Yu et al.~\autocite{yu2024humanevalpro}, modify existing problems to require multi-step reasoning, with intermediate outputs feeding subsequent subtasks. Results show significant accuracy drops even in advanced models, highlighting limitations in compositional problem solving. Meanwhile, CommitBench~\autocite{Schall2024CommitBench} evaluates natural-language generation for commit messages from code diffs, offering insight into adjacent software engineering tasks beyond code synthesis.

Together, these new datasets reflect a growing focus on robustness, generalization, and real-world usability. They challenge models beyond surface-level correctness and point toward richer protocols for evaluating assistance across the full software development lifecycle.

% \clearpage

\section{Summary}
\label{sec:summary}
The integration of \gls{AI} into software development has progressed rapidly over the past few years. Pre-AI tools such as linters, static analyzers, and rule-based autocompletion offered deterministic, non-generative support but remained limited to pattern recognition and diagnostics. The emergence of deep learning and transformer-based models, beginning with early prototypes like TabNine and culminating in commercial tools like Copilot, introduced a transformative shift in developer workflows. These \glspl{LLM} enabled generative capabilities, allowing developers to write code from natural language prompts and significantly accelerating routine development tasks. However, widespread adoption has raised concerns about security, licensing compliance, and consequently, legal risk. Models have been shown to produce insecure or plagiarized code, and organizations have responded with growing interest in self-hosted alternatives. Cases like Samsung’s internal ban on ChatGPT and JPMorgan’s development of an in-house \gls{LLM} platform illustrate this shift toward privacy-preserving solutions. The release of open-weight models like LLaMA 2 and StarCoder has made it increasingly viable to deploy powerful coding assistants locally, provided proper infrastructure and tuning are in place. To evaluate such models in controlled, secure environments, rigorous benchmarks are essential. Execution-based datasets like HumanEval, \gls{MBPP}, and \gls{APPS} allow for functional assessment of code generation, while benchmarks like DS-1000  and MultiPL-E explore real-world and multilingual capabilities. These evaluations inform model selection and guide fine-tuning efforts.




% Voeg hier je eigen hoofdstukken toe die de ``corpus'' van je bachelorproef
% vormen. De structuur en titels hangen af van je eigen onderzoek. Je kan bv.
% elke fase in je onderzoek in een apart hoofdstuk bespreken.

%\input{...}
%\input{...}
%...
%-------------------------------------------------------------------------%
%  Proof-of-Concept Overview                                              %
%-------------------------------------------------------------------------%
\chapter{Proof of Concept}
\label{ch:proof-of-concept}

In this proof of concept, we present a structured evaluation of modern self-hosted language models for code generation, motivated by growing enterprise demand for privacy-preserving developer tools. The study compares leading open-weight models---Qwen2.5 and Codestral-in a containerized, fully reproducible benchmarking pipeline, and assesses their viability as on-prem alternatives to proprietary cloud-based solutions such as GitHub Copilot or ChatGPT. Emphasis is placed not only on accuracy (via functional correctness benchmarks), but also on practical deployment characteristics: licensing, resource demands, inference flexibility, and output format compatibility. The end-to-end pipeline is built around the \texttt{bigcode-evaluation-harness}, with all experiments conducted inside a GPU-accelerated Docker container to ensure isolation and determinism. %Technical setup details, including ROCm support and hardware tuning, are documented separately in Appendix~A. The following sections justify our platform choice, model selection, and benchmarking design.

%=========================================================================%
\section{Justification of Approach}

%-------------------------------------------------------------------------%
\subsection{Model Deployment}

%====================  MODEL DEPLOYMENT PLATFORM  ====================%

To support the deployment of locally hosted language models, we evaluated two viable platforms: the Hugging Face Transformers library, an industry-standard toolkit for loading, configuring, and running open-weight language models via Python and the Ollama runtime, a lightweight framework designed for launching \glspl{LLM} through a simplified CLI-based workflow. Both platforms satisfy a foundational requirement of our project---offline inference with no data leaving the local environment---making them suitable for privacy-conscious developer use cases. However, they represent fundamentally different approaches to deployment: HuggingFace emphasizes flexibility and fine-grained control through programmatic interfaces, while Ollama prioritizes simplicity and accessibility. Given our need to automate large-scale benchmarking and ensure full transparency and reproducibility of runs, it was essential to weigh each platform’s strengths and limitations systematically. Table~\ref{tab:platform-checklist} summarizes this comparative evaluation across a set of practical criteria, each reflecting a technical or operational requirement relevant to our research objectives.

\begin{table}[ht]
	\centering
	\caption{Checklist of requirements for local code-model deployment.}
	\label{tab:platform-checklist}

	\renewcommand{\arraystretch}{1.2}      % row height
	% One *single* tabularx:
	\begin{tabularx}{\textwidth}{@{}X
		>{\centering\arraybackslash}p{3cm}
		>{\centering\arraybackslash}p{3cm}@{}}
		\toprule
		\textbf{Requirement}                       & \textbf{Hugging Face} & \textbf{Ollama} \\
		\midrule
		Native access to the latest models         & \checkYes             & \checkNo        \\
		Run natively on AMD ROCm (via PyTorch)     & \checkYes             & \checkNo        \\
		Fine-grained control of inference settings & \checkYes             & \checkNo        \\
		Deterministic output and version pinning   & \checkYes             & \checkNo        \\
		Programmatically parsable output           & \checkYes             & \checkNo        \\
		Minimal install effort                     & \checkYes             & \checkYes       \\
		\bottomrule
	\end{tabularx}
\end{table}


%====================== JUSTIFICATION PER REQUIREMENT ======================%
\paragraph{Native access to the latest models}
One of the key differences between Hugging Face Transformers and Ollama lies in their respective access to state-of-the-art model checkpoints. Hugging Face provides a vast and actively maintained model hub, where new releases from both research institutions and commercial developers are published frequently---often within days of their official announcement. This includes all major models discussed in our literature study---such as StarCoder, CodeLlama, and WizardCoder---which are readily available in their original formats (e.g., FP16, BF16) without the need for additional conversion or repackaging. In contrast, Ollama requires that models be pre-converted into the GGUF format (used by llama.cpp) and hosted in its own limited registry. While community conversions do exist, they often lag behind the latest model releases and may not include metadata essential for accurate inference (e.g., tokenizer vocab alignment, EOS token placement). This friction disqualifies Ollama as a primary platform in research settings where keeping up with the cutting edge is a prerequisite.

\paragraph{Run natively on AMD ROCm (via PyTorch)}
Given our compute environment includes an AMD Radeon GPU, native ROCm support was a non-negotiable requirement. Hugging Face Transformers, when paired with PyTorch ROCm builds, enables native GPU acceleration under Linux and WSL2, unlocking hardware-specific features like bfloat16 tensor cores and multi-stream compute. This is crucial not only for benchmarking performance (latency, throughput) but also for ensuring that model behavior under accelerated conditions remains reproducible. Ollama, in contrast, builds upon llama.cpp, which---at the time of writing---lacks official ROCm support and primarily targets NVIDIA (via CUDA) and CPU execution. Although some community forks have experimented with HIP integration, these are unofficial and unstable, making Ollama unsuitable for consistent ROCm-based deployments. As a result, only Hugging Face fulfilled this requirement out of the box.

\paragraph{Fine-grained control of inference settings}
Effective evaluation of code-generation models hinges on the ability to configure decoding parameters with precision. Parameters such as sampling temperature, top-$k$ and top-$p$ filtering, repetition penalties, maximum generation length, and custom stop sequences can substantially influence output quality, particularly on constrained tasks like HumanEval. The use of custom stop sequences is especially critical when working with instruction-tuned models, which often append additional text (e.g., closing prompts or signatures) unless explicitly instructed to halt generation. Hugging Face Transformers provides comprehensive access to these settings through its generation API, facilitating consistent control over generation behavior across evaluations. This level of granularity is indispensable for achieving reproducible results and conducting ablation studies that isolate the impact of specific inference strategies.

By contrast, Ollama adopts a more opaque approach to inference. While it supports a limited subset of parameters through command-line flags or its HTTP interface, many advanced generation settings are either unavailable or difficult to modify dynamically. This limitation creates a significant barrier for automated evaluation workflows that depend on programmatic control of decoding configuration---such as those implemented in benchmark frameworks like \texttt{lm-evaluation-harness} or \texttt{bigcode-evaluation-harness} (Section~\ref{sec:pipeline}). Since these frameworks rely on direct integration with libraries like \texttt{transformers} to orchestrate controlled batch evaluation, Ollama’s architecture introduces compatibility issues that hinder its applicability in rigorous benchmarking contexts. The absence of standard interfaces for setting inference parameters at runtime further complicates integration with reproducibility-focused pipelines and limits its utility in comparative experiments.

\paragraph{Deterministic output and version pinning}
To make our benchmarking process scientifically valid, we needed all results to be both deterministic and attributable to a specific software stack. Hugging Face enables this by allowing explicit pinning of library versions---e.g., \texttt{transformers==4.39.3}, \texttt{torch==2.2.1+rocm}-and by supporting fixed random seeds for all generation calls. When these controls are in place, identical inputs yield identical outputs across different machines, assuming the same compute backend. This is critical when comparing models across multiple benchmarks, or when regenerating outputs post-review. Ollama, being more of a “black box,” hides many of its internal dependencies and updates its containerized runtimes automatically in the background. Although this makes for a smoother end-user experience, it reduces transparency and risks breaking reproducibility between test runs---particularly problematic in research or regulated enterprise environments.

\paragraph{Programmatically parsable output}
For seamless integration with our benchmarking infrastructure, it was imperative that generated outputs be captured in a structured and machine-readable format. Hugging Face allows programmatic access to raw model outputs via Python objects, making it trivial to format completions into JSON lines or pipe them directly into downstream evaluation tools like \texttt{lm-evaluation-harness} and \texttt{bigcode-evaluation-harness} (Section~\ref{sec:pipeline}). This design supports filtering, tagging, and aggregating completions with metadata like prompt IDs or timestamped seeds. Ollama, however, defaults to streaming its outputs to STDOUT---designed for interactive command-line use rather than batch processing. While it is possible to capture and post-process this output, doing so introduces brittle shell-based parsing or requires the use of its REST API, which lacks standardization across model versions. For our purposes, Hugging Face was the only platform that offered the level of integration and control necessary for a scalable and automatable evaluation pipeline.

\paragraph{Minimal install effort}
Despite its complexity and configurability, Hugging Face remains relatively lightweight in terms of setup: installing \texttt{transformers}, \texttt{torch}, and a few supporting packages via pip is sufficient to run most models. For users already familiar with Python-based environments and virtual environments, this installation process is both predictable and easily scripted for Docker-based deployment. Ollama, by contrast, aims to minimize all configuration overhead. It can be installed with a single command and invoked immediately without writing a single line of code. From a usability standpoint, this is highly commendable, and for non-technical users or quick local tests, Ollama excels. However, in our case, since we already required containerization, scripting, and integration with evaluation tooling, the additional install effort associated with Hugging Face did not pose a disadvantage and was offset by its flexibility.

\paragraph{Conclusion}
While Ollama excels at simplicity and quick experimentation, its design targets ease-of-use over configurability. For our purposes---namely, rigorous evaluation of code generation models in a reproducible, research-grade setting---it proved insufficient. Key shortcomings included its limited control over inference parameters, lack of support for deterministic runs via version pinning, and the absence of direct compatibility with established evaluation pipelines such as \texttt{lm-evaluation-harness} and \texttt{bigcode-evaluation-harness}. Moreover, Ollama's opaque runtime and console-based output make it difficult to extract structured data needed for automated benchmarking. In contrast, the Hugging Face Transformers library offered a far more comprehensive solution: full access to the latest model checkpoints, seamless integration with our tooling, and fine-grained control over every aspect of inference. These features made it not just a more flexible platform, but the only viable choice for ensuring that our findings could be reproduced, extended, and trusted within an academic or enterprise context.

%-------------------------------------------------------------------------%
%  Model Selection                                                         %
%-------------------------------------------------------------------------%
\subsection{Model selection}
\label{subsection:model-selection}
In order to accurately assess the current state of locally hosted, open-weight language models for code generation, we established a clear and pragmatic selection strategy. This strategy was developed in close collaboration with the co-supervisor of this thesis, Mr.~Leflon, whose professional background in cybersecurity informed the filtering process for secure enterprise deployment. Drawing on his industry insights, we formulated a concrete set of requirements (see Table~\ref{tab:model-reqs}) that reflect the operational needs and constraints typical of real-world on-prem environments. These include licensing conditions that allow unrestricted commercial use, parameter limits for single-GPU inference, compatibility with quantization workflows, long context window support, architectural variety, and sufficient performance on practical code generation tasks.

The goal was not to exhaustively sample all available open-source models, but to identify a diverse yet representative subset of 7B-scale models that exemplify current best practices in open-access LLM development. Many open-weight code models were reviewed and ultimately excluded---either due to outdated architecture, insufficient benchmark performance, limited community support, or restrictive licensing. Notably, we rejected WizardCoder (due to a lack of strong 7B variants), Magicoder (largely derivative of older CodeLlama backbones), and OpenHermes-Code (minimal evaluation and ecosystem traction). Smaller models like Replit Code-v1.5 were excluded based on sub-threshold parameter count and constrained capability. Similarly, general-purpose LLaMA 2 variants and forks like CodeUp or OpenChat-Code were not considered competitive with more recent instruction-tuned or code-specialized models released in 2023-2024.

This curated list ensures that our evaluation remains forward-looking, reproducible, and aligned with the practical demands of deploying secure, in-house language models. Moreover, by establishing a common benchmark foundation, we can compare pass@1 performance across the cohort and perform a fair preliminary ranking to further narrow down candidates for deeper evaluation in downstream tasks.

\begin{table}[H]
	\centering
	\caption{Objective requirements for selecting an on-prem code model.}
	\label{tab:model-reqs}
	\renewcommand{\arraystretch}{1.2}

	%--------------------------------------------------------------------%
	\begin{tabularx}{\textwidth}{@{}>{\centering\arraybackslash}p{0.8em}%
		>{\raggedright\arraybackslash}p{3.8cm}%
		X@{}}
		\toprule
		\textbf{\#} & \textbf{Requirement}                          & \textbf{Why it matters for an on-prem assistant}                                    \\
		\midrule
		1           & Permissive licence                            & Allows commercial deployment and avoids legal ambiguity                             \\[2pt]
		2           & \leq{}7B parameters                           & Enables single-GPU inference (\leq{}20GB VRAM)                                      \\[2pt]
		3           & Instruction-tuned                             & Supports natural language prompts, essential for code assistance use cases          \\[2pt]
		4           & Minimum context window size (\geq{}4k tokens) & Supports single-file and function-level code completion and moderately long prompts \\[2pt]
		5           & Quantisation-ready                            & Supports 4-8-bit precision for laptop/server deployment                             \\[2pt]
		6           & Competitive pass@1                            & Ensures sufficient real-world coding competence                                     \\[2pt]
		7           & Architectural diversity                       & Reduces overfitting to one model family; tests generalizability                     \\[2pt]
		8           & Active community                              & Ensures ongoing updates, support, and fine-tuning recipes                           \\[2pt]
		\bottomrule
	\end{tabularx}
\end{table}

After filtering with these criteria, we selected eight models that together form a rigorous, architecture-diverse benchmark cohort: \texttt{StarCoder2-7B}, \texttt{CodeGemma-7B-Instruct}, \texttt{CodeLlama-7B-Python}, \texttt{Qwen2.5-Coder-7B-Instruct}, \texttt{DeepSeek-Coder-7B-Instruct-v1.5}, \texttt{Mistral-7B-Instruct-v0.3}, and \texttt{Mamba-Codestral-7B-v0.1}. These models span different pretraining strategies, base architectures (e.g., LLaMA, Mistral, Gemma, Qwen), and instruction-tuning approaches. All of them support quantization, are self-hostable, and meet the resource constraints defined for single-GPU deployment in enterprise settings.

\paragraph{Permissive license}
All selected models are released under licenses that permit self-hosted commercial deployment, a strict requirement for enterprise integration. Most---such as \texttt{Qwen2.5-Coder-7B-Instruct}, \texttt{Mamba-Codestral-7B}, \texttt{Mistral-7B-Instruct-v0.3}, and \texttt{DeepSeek-Coder-7B-Instruct-v1.5}---are covered by permissive open-source licenses such as Apache 2.0 or MIT, which allow modification, redistribution, and private usage without complex legal barriers. Although \texttt{CodeLlama-7B-Python} and \texttt{CodeGemma-7B-Instruct} operate under more restrictive terms (Meta’s Llama 2 Community License and Google’s Gemma License, respectively), both explicitly allow internal commercial use and model hosting, which satisfies the legal needs of on-prem enterprise workflows.

\paragraph{\(\leq\)7B parameters}
All eight models contain approximately 7 billion parameters, a scale that maximizes the tradeoff between performance and resource efficiency. This ensures compatibility with single-GPU setups equipped with high-memory consumer or workstation-grade GPUs (e.g., NVIDIA RTX 3000/4000/5000-series or AMD RX 7900 XT/XTX), enabling local deployment even in mid-sized organizations without data center-level infrastructure---an explicit requirement derived from the practical constraints of this thesis's co-supervisor’s operational environment. Models exceeding this size, such as WizardCoder-15B or Falcon-40B, were excluded due to their prohibitive memory footprint, which is incompatible with our single-GPU requirement.

\paragraph{Instruction-tuned}
Each selected model is explicitly instruction-tuned for conversational or code-assistant use cases. \texttt{Mamba-Codestral-7B}, \texttt{Qwen2.5-Coder-7B-Instruct}, \texttt{DeepSeek-Coder-7B-Instruct-v1.5}, and \texttt{CodeGemma-7B-Instruct} have undergone supervised fine-tuning on high-quality code tasks framed as natural language prompts. \texttt{Mistral-7B-Instruct-v0.3} generalizes instruction-following for a broad range of reasoning and logic-based tasks. This tuning enables models to reliably respond to developer queries---whether generating function bodies, rewriting legacy code, or providing inline explanations---without requiring prompt engineering or templating.

\paragraph{Minimum context window}
Several models in our selection exceed the baseline 4\,K token window required for file‐level tasks: \texttt{Mamba-Codestral-7B} supports up to a remarkable 256\,K tokens; \texttt{Qwen2.5-Coder-7B-Instruct} provides 128\,K tokens, enabling advanced cross‐file retrieval workflows; \texttt{Mistral-7B-Instruct-v0.3} extends to 32\,K tokens; and \texttt{StarCoder2-7B} offers 16\,K via sliding‐attention, ideal for mid‐sized source files. In contrast, \texttt{CodeLlama-7B-Python} and \texttt{CodeGemma-7B-Instruct} maintain the standard 8\,K token window, while \texttt{DeepSeek-Coder-7B-Instruct-v1.5} remains at 4\,K tokens, perfectly suited to single‐file or function‐level completions. A minimum 4\,K token context is critical for unit‐test generation, file‐wide refactoring, and retrieval‐augmented generation in enterprise codebases.

\paragraph{Quantization-ready}
All models are available or compatible with 4-bit and 8-bit quantized formats, including GGUF and \texttt{bitsandbytes} variants. Quantization allows real-time inference on consumer hardware and significantly reduces memory and compute costs. Community-driven tooling ensures that models like \texttt{StarCoder2-7B}, \texttt{Qwen2.5}, and \texttt{Mistral-7B-Instruct} are easily portable. This deployment flexibility is critical for organizations that must operate in secure environments---such as air-gapped systems with no external network access, or on remote edge devices requiring fully local inference.

\paragraph{Competitive pass@1}
The selected models demonstrate state-of-the-art performance on HumanEval, MBPP, or equivalent code benchmarks. \texttt{Mamba-Codestral-7B} exceeds 70\% pass@1 on HumanEval, while \texttt{Qwen2.5-Coder-7B-Instruct} and \texttt{DeepSeek-Coder-7B-Instruct-v1.5} achieve competitive results often surpassing older LLaMA-based baselines. \texttt{CodeGemma-7B-Instruct} and \texttt{CodeLlama-7B-Python} perform well in Python-centric benchmarks, which remain central in enterprise automation and data workflows. Pass@1 performance was a minimum bar to inclusion, as it reflects functional correctness without the need for excessive sampling or reranking.

\paragraph{Architectural diversity}
The final cohort includes models from five distinct architectural lineages: LLaMA-based models (\texttt{CodeLlama}, \texttt{Mistral-7B}, DeepSeek variants), the Qwen transformer family (\texttt{Qwen2.5}), Google’s Gemma series (\texttt{CodeGemma}), and the Mamba linear attention architecture (\texttt{Mamba-Codestral}). This variety ensures our evaluation captures behavior across differing design philosophies, positional encodings, tokenization strategies, and attention mechanisms. It also mitigates bias in task performance and improves the generalizability of our findings to future models.

\paragraph{Active community}
Every selected model benefits from significant community activity, including ongoing development, quantization forks, training insights, and issue tracking. \texttt{Qwen2.5} and \texttt{Mistral-7B} are widely adopted across fine-tuning pipelines; \texttt{StarCoder2} is supported by BigCode’s open community; \texttt{DeepSeek} models are backed by frequent updates; and \texttt{Mamba-Codestral} benefits from Mistral’s rapid release cadence. In contrast, excluded models like OpenHermes or Magicoder lack sustained development or reproducibility guarantees. Community engagement ensures long-term viability, which is critical for internal enterprise tooling.

\paragraph{Conclusion}
Together, the eight selected models comprehensively satisfy the requirements defined in Table~\ref{tab:model-reqs}. Their licensing terms, compute footprint, instruction-following capabilities, benchmark performance, and support ecosystems make them ideal candidates for secure, self-hosted code generation in professional software environments. This curated benchmark group enables us to evaluate how far open-weight models have progressed in meeting the needs of real-world enterprise development workflows.
%-------------------------------------------------------------------------%
%  Evaluation Pipeline (lm-evaluation-harness)                            %
%-------------------------------------------------------------------------%
\subsection{Evaluation Pipeline}
\label{sec:pipeline}

To evaluate the models outlined in~\ref{subsection:model-selection} on code-generation benchmarks, we leverage the \texttt{bigcode-evaluation-harness}, a benchmarking suite developed by the BigCode project specifically for assessing code-oriented language models \autocite{Bigcode}. This framework extends the more general-purpose \texttt{lm-evaluation-harness} created by EleutherAI~\autocite{EleutherAI}, which offers limited support for code-related benchmarks such as HumanEval and MBPP, but is primarily designed for \gls{NLP} tasks like MMLU, GSM8K, and HellaSwag. While both harnesses are extensible, the BigCode implementation provides out-of-the-box support for a broader set of code-specific tasks---including HumanEval+, HumanEval Instruct, DS-1000, and APPS---using test-suite execution and function-level correctness as core evaluation metrics. These metrics align closely with the practical demands of software development. Although EleutherAI’s framework can be extended to support such tasks, doing so would require significant engineering effort to replicate the domain-specific evaluation logic and test infrastructure already built into the BigCode harness.

Both harnesses are designed to interface seamlessly with Hugging Face Transformers, but the BigCode evaluation suite introduces additional infrastructure tailored to code-centric workflows. Notably, it includes preconfigured stop sequences for consistent code formatting, execution scaffolds for running test cases, and secure evaluation tools that mitigate risks such as arbitrary code execution. These capabilities are essential when assessing instruction-tuned code models, where incomplete or overly verbose generations can distort evaluation outcomes. Furthermore, the BigCode harness integrates natively with the \texttt{accelerate} library to facilitate multi-GPU inference and includes containerized deployment recipes for streamlined experimentation. By contrast, while EleutherAI’s \texttt{lm-evaluation-harness} supports local and API-based execution of Hugging Face models, its focus remains on general NLP tasks and it lacks the specialized defaults and performance-oriented enhancements that are critical for rigorous evaluation of code generation.

Importantly, both frameworks support key configuration controls such as fixed random seeds, adjustable sampling strategies (e.g., temperature, top-$k$, top-$p$), and output logging. However, \texttt{bigcode-evaluation-harness} streamlines reproducibility by encapsulating these settings within a unified evaluation interface and aligning outputs with the expected schema for each benchmark. It also produces structured JSON artifacts---including model generations, reference completions, and scoring results---in a consistent, machine-readable format, which facilitates downstream analysis and integration with custom evaluation pipelines.

A sample CLI usage of \texttt{bigcode-evaluation-harness} is shown below:

\begin{listing}[H]
	\begin{minted}[linenos, breaklines, fontsize=\small]{bash}    
    accelerate launch main.py \
        --model <model_name> \
        --precision bf16 \
        --tasks humaneval,humanevalplus,mbpp,mbppplus \
        --do_sample False \
        --n_samples 1 \
        --batch_size 1 \
        --save_generations \
        --save_references \
        --allow_code_execution

\end{minted}
	\caption{Example usage of the BigCode evaluation harness to evaluate multiple models.}
	\label{lst:bceh}
\end{listing}

This command generates code samples for each benchmark task, executes unit tests, and logs results for each model configuration.

In summary, while both \texttt{bigcode-evaluation-harness} and \texttt{lm-evaluation-harness} offer extensible benchmarking infrastructure, \texttt{bigcode-evaluation-harness} provides a significantly lower barrier to evaluating code models due to its native support for unit-test-based code tasks, integration with code-specific formatting and execution settings, and compatibility with local deployment. Its self-contained design and focus on reproducibility made it the more appropriate choice for our privacy-sensitive, on-premise benchmarking pipeline.

%-------------------------------------------------------------------------%
%  Benchmark Selection                                                    %
%-------------------------------------------------------------------------%
\subsection{Benchmark Selection}
\label{sec:benchmarks}

A core objective of this study is to evaluate locally hosted \glspl{LLM} in their role as coding assistants, specifically in generating correct and usable code snippets for developer tasks. To support this goal, we prioritized benchmarks that assess functional correctness on well-scoped programming problems, rather than measuring broader capabilities like full program synthesis or documentation generation. Based on our literature study (Section~\ref{sec:llm-benchmarks}) and the set of officially supported tasks in the \texttt{bigcode-evaluation-harness}, we selected both the original HumanEval and \gls{MBPP} benchmarks, as well as their respective augmented variants: HumanEval+ and \gls{MBPP}+, developed under the EvalPlus initiative~\autocite{Liu2023Rigorous}. Together, these four benchmarks provide a layered evaluation strategy that balances legacy comparability with modern diagnostic depth.

\paragraph{HumanEval and HumanEval+}
HumanEval is a widely used benchmark comprising 164 manually written Python problems~\autocite{Chen2021}. Each task consists of a function signature, a descriptive docstring, and a hidden suite of unit tests for automatic correctness validation. It provides a compact but rigorous setting for evaluating first-attempt code synthesis—an important requirement for offline deployments, where iterative sampling may not be feasible.

However, as highlighted in recent literature~\autocite{Liu2023Rigorous}, HumanEval’s limited number of test cases per task can lead to inflated pass@1 scores, particularly for models prone to overfitting on surface-level patterns. To address this, we also adopted HumanEval+, which preserves the original prompts but augments each with over 80 unit tests generated via a combination of LLM-based synthesis and mutation strategies. This significantly strengthens the reliability of performance metrics, especially in scenarios demanding real-world robustness.

\paragraph{MBPP and MBPP+}
The \gls{MBPP} benchmark extends the evaluation landscape with 974 crowd-sourced Python problems, each framed as a natural-language instruction and paired with hidden test cases~\autocite{Austin2021}. MBPP’s prompts more closely resemble how developers describe problems in practice, making it especially relevant for assessing instruction adherence and contextual interpretation in coding assistants.

To mitigate concerns about narrow test coverage, we included MBPP+, an augmented subset of 399 tasks selected for functional clarity and extended with significantly larger test suites~\autocite{Liu2023Rigorous}. The augmented test cases enforce correctness under a more diverse set of inputs and edge cases while preserving MBPP’s original problem format. As with HumanEval+, this allows us to probe not only functional correctness, but also prompt stability—whether a model’s behavior remains consistent when task phrasing or test inputs vary.

\subsubsection*{Summary}
Together, HumanEval and MBPP offer historical continuity, linguistic diversity, and problem-scale variation, while HumanEval+ and MBPP+ improve diagnostic resolution through denser test coverage. Evaluating models across both base and augmented variants allows us to distinguish between models that are merely prompt-aligned and those that generalize reliably. This combination is particularly suited to our investigation of secure, on-premise deployment, where first-pass correctness and robustness to prompt variability are critical performance attributes. By integrating EvalPlus into our evaluation, we provide a more stringent and realistic benchmark framework aligned with enterprise coding assistant requirements.

\clearpage

\section{Practical Setup and Execution}

To validate our benchmarking strategy for evaluating instruction-tuned code generation models, we implemented a containerized, GPU-accelerated pipeline built around the BigCode evaluation harness. Our implementation targets execution on ROCm-compatible AMD GPUs running under WSL2, leveraging local model caching for reproducibility and performance. Below are the key steps:

\subsection{Dockerfile}

\begin{minipage}{\textwidth}
	\captionsetup{type=listing}
	\caption{Optimised \texttt{Dockerfile} for ROCm-based benchmarking container}
	\label{lst:dockerfile}

	\begin{minted}[linenos,
                    breaklines,
                    breakanywhere=true,
                    fontsize=\footnotesize]{docker}
                    ###############################################################################
                    #  Ubuntu 24.04 · ROCm 6.3.4 · PyTorch 2.4 · BigCode
                    ###############################################################################
                    FROM ubuntu:noble
                    LABEL maintainer="Elias Csoregh <elias.csoregh@hogent.student.be>"

                    ARG  ROCM_VER=6.3.4
                    ARG  TORCH_VER=2.4.0

                    ENV  DEBIAN_FRONTEND=noninteractive
                    ENV  PATH=/opt/venv/bin:$PATH
                    WORKDIR /tmp

    \end{minted}
\end{minipage}

This initial block defines the foundational build instructions for the container. The \texttt{FROM ubuntu:24.04} directive specifies the base image, selecting \texttt{Ubuntu 24.04} for its long-term support and compatibility with the \texttt{ROCm 6.3.4} release.

The use of \texttt{ARG} is employed to explicitly pin the versions of critical dependencies, namely \texttt{ROCm} and \texttt{PyTorch}, via the variables \texttt{ROCM\_VER} and \texttt{TORCH\_VER}. This ensures that all downstream installation steps rely on versions that are known to be mutually compatible. While \texttt{ARG} also enables modular reuse, in this context it primarily serves to enforce version consistency and reproducibility across builds --- a best practice in containerized environments dealing with hardware-accelerated libraries.

Subsequently, the \texttt{ENV} instructions define environment variables within the container runtime. Setting \texttt{DEBIAN\_FRONTEND=noninteractive} ensures that APT operations proceed without prompting for user interaction, which is essential in non-interactive Docker builds. The \texttt{PATH=/opt/venv/bin:\$PATH} declaration prepends the virtual environment path, ensuring that any installed Python binaries or tools are prioritized during execution. This is important for maintaining isolation from system-level dependencies.

Finally, the working directory is set to \texttt{/tmp} via \texttt{WORKDIR}, establishing a neutral staging area for all initial setup operations, such as downloading packages and building local wheel caches.

\vspace{1em}

\begin{minipage}{\textwidth}
	\captionsetup{type=listing}
	\caption*{Listing~\ref{lst:dockerfile}~(continued)}

	\begin{minted}[firstnumber=14,
                    breaklines,
                    breakanywhere=true,
                    fontsize=\footnotesize]{docker}
                    # ───────────────────────── 0. basic packages + venv ──────────────────────────
                    RUN apt-get update && \
                        apt-get install -y --no-install-recommends \
                            python3-full python3-venv python3-pip \
                            git wget ca-certificates rsync dialog && \
                        rm -rf /var/lib/apt/lists/*

                    RUN python3 -m venv /opt/venv && \
                        pip install --no-cache-dir --upgrade pip wheel

    \end{minted}
\end{minipage}

This block installs the essential system-level packages required to support Python development and container orchestration. The process begins with \texttt{apt-get update}, which refreshes the package index to ensure the latest metadata is available for installation. This is followed by a carefully scoped \texttt{apt-get install} command that uses the \texttt{--no-install-recommends} flag to suppress installation of non-essential packages, thereby reducing image bloat and limiting the build to explicitly specified dependencies.

The selected packages include:

\begin{itemize}
	\item \texttt{python3-full}, \texttt{python3-venv}, and \texttt{python3-pip}: These provide access to Python's full standard library, support for creating isolated virtual environments, and the package installer \texttt{pip}, respectively. The inclusion of \texttt{python3-full} ensures compatibility with scripts or libraries that may depend on optional modules not present in minimal distributions.
	\item \texttt{git} and \texttt{wget}: Required for cloning external repositories and downloading resources over HTTP or HTTPS. These are essential for fetching the \texttt{BigCode} evaluation harness and prebuilt \texttt{PyTorch} wheels from AMD's repositories.
	\item \texttt{ca-certificates}: Ensures the system trusts root certificate authorities, which is critical for securely accessing external services via HTTPS (e.g., Hugging Face or GitHub).
	\item \texttt{rsync} and \texttt{dialog}: Lightweight utilities that improve scripting flexibility and are occasionally required by package installers.
\end{itemize}

Once the necessary packages are installed, \texttt{rm -rf /var/lib/apt/lists/*} is executed to remove cached package metadata. This further reduces the final image size and prevents stale index data from persisting across layers.

The second \texttt{RUN} command initializes a virtual environment in \texttt{/opt/venv} using \texttt{python3 -m venv}. This isolates all subsequent Python package installations from the system Python interpreter. Immediately after, \texttt{pip} and \texttt{wheel} are upgraded to their latest available versions to ensure compatibility with modern packaging standards. This is especially important when installing packages that use binary wheel formats or require PEP 517/518-based builds.

Together, these instructions establish a minimal, reproducible, and robust foundation for Python-based evaluation workflows within the containerized environment.

\vspace{1em}

\begin{minipage}{\textwidth}
	\captionsetup{type=listing}
	\caption*{Listing~\ref{lst:dockerfile}~(continued)}

	\begin{minted}[firstnumber=24,
                    breaklines,
                    breakanywhere=true,
                    fontsize=\footnotesize]{docker}
                    # ───────────────────────── 1. ROCm meta-package (WSL, no-DKMS) ───────────────
                    WORKDIR /tmp
                    RUN wget -q https://repo.radeon.com/amdgpu-install/6.3.4/ubuntu/noble/amdgpu-install_6.3.60304-1_all.deb && \
                        apt-get update && \
                        apt-get install -y ./amdgpu-install_6.3.60304-1_all.deb && \
                        rm -f amdgpu-install_6.3.60304-1_all.deb && \
                        amdgpu-install -y --usecase=wsl,rocm --no-dkms && \
                        rm -rf /var/lib/apt/lists/*

    \end{minted}
\end{minipage}

This block installs the ROCm base stack required for enabling GPU-accelerated computation on AMD hardware within the container. The working directory is first set to \texttt{/tmp} using \texttt{WORKDIR}, which serves as a temporary staging area for downloading and installing intermediate assets.

The main task here is to install the \texttt{amdgpu-install} meta-package provided by AMD, which orchestrates the configuration of ROCm components. The installation begins by retrieving the Debian package \texttt{amdgpu-install\_6.3.60304-1\_all.deb} via \texttt{wget}. This specific version corresponds to \texttt{ROCm 6.3.4} and is hosted on AMD’s official package repository for Ubuntu \texttt{noble}.

After refreshing the package index with \texttt{apt-get update}, the downloaded package is installed using \texttt{apt-get install}. Immediately afterward, the \texttt{.deb} file is removed via \texttt{rm -f} to conserve disk space and avoid leaving unnecessary installation artifacts in the image.

The command \texttt{amdgpu-install -y --usecase=wsl,rocm --no-dkms} is then executed to install only the components relevant to ROCm and Windows Subsystem for Linux (WSL), while deliberately omitting kernel module installation via the \texttt{--no-dkms} flag. This is critical: since WSL2 lacks full kernel support, attempting to build or load DKMS (Dynamic Kernel Module Support) drivers would fail or cause instability. The use case specification \texttt{wsl,rocm} ensures that only userspace ROCm libraries are installed---those necessary for leveraging GPU compute inside the container without relying on kernel-level drivers.

Finally, the block concludes with a standard cleanup command \texttt{rm -rf /var/lib/apt/lists/*}, which removes cached APT metadata and further reduces the image size, helping maintain a lightweight and reproducible container.

\vspace{1em}

\begin{minipage}{\textwidth}
	\captionsetup{type=listing}
	\caption*{Listing~\ref{lst:dockerfile}~(continued)}

	\begin{minted}[firstnumber=33,
                    breaklines,
                    breakanywhere=true,
                    fontsize=\footnotesize]{docker}
                    # ───────────────────────── 2. ROCm PyTorch wheels ────────────────────────────
                    WORKDIR /tmp/wheels
                    RUN set -e; base=https://repo.radeon.com/rocm/manylinux/rocm-rel-6.3.4; \
                        wget -q ${base}/torch-2.4.0%2Brocm6.3.4.git7cecbf6d-cp312-cp312-linux_x86_64.whl && \
                        wget -q ${base}/torchvision-0.19.0%2Brocm6.3.4.gitfab84886-cp312-cp312-linux_x86_64.whl && \
                        wget -q ${base}/pytorch_triton_rocm-3.0.0%2Brocm6.3.4.git75cc27c2-cp312-cp312-linux_x86_64.whl && \
                        wget -q ${base}/torchaudio-2.4.0%2Brocm6.3.4.git69d40773-cp312-cp312-linux_x86_64.whl && \
                        pip uninstall -y torch torchvision torchaudio pytorch-triton-rocm || true && \
                        pip install --no-cache-dir *.whl && \
                        rm -rf /tmp/wheels

                    # remove conflicting lib bundled with torch
                    RUN python - <<'PY'
                    import torch, pathlib, glob, os
                    libdir = pathlib.Path(torch.__file__).parent / "lib"
                    for so in glob.glob(str(libdir / 'libhsa-runtime64.so*')):
                        print('Removing', so); os.remove(so)
                    PY

    \end{minted}
\end{minipage}

This stage prepares the environment for installing ROCm-compatible \texttt{PyTorch} packages that are officially provided by AMD for use with the \texttt{ROCm 6.3.4} release. The working directory is first set to \texttt{/tmp/wheels}, which serves as a temporary location for downloading the wheel files prior to installation.

The \texttt{RUN} instruction begins with \texttt{set -e}, a shell flag that causes the command sequence to terminate immediately if any sub-command fails. This ensures strict build correctness and avoids partial or corrupted installations.

A base URL is defined as a shell variable (\url{https://repo.radeon.com/rocm/manylinux/rocm-rel-6.3.4}), pointing to AMD’s ROCm-specific Python wheel repository. From this repository, four distinct packages are retrieved using \texttt{wget}:

\begin{itemize}
	\item \texttt{torch-2.4.0+rocm6.3.4} - the core \texttt{PyTorch} library compiled with ROCm backend support.
	\item \texttt{torchvision-0.19.0+rocm6.3.4} - utilities for image transformation and dataset handling.
	\item \texttt{pytorch\_triton\_rocm-3.0.0+rocm6.3.4} - a ROCm-compatible build of \texttt{Triton}, supporting fused GPU kernel execution.
	\item \texttt{torchaudio-2.4.0+rocm6.3.4} - the \texttt{PyTorch} audio library with ROCm backend compatibility.
\end{itemize}

Any previously installed versions of these packages (including possible incompatible system defaults) are removed using \texttt{pip uninstall -y}, and the new ROCm-specific wheels are installed via \texttt{pip install --no-cache-dir *.whl}. Finally, the temporary wheel directory is deleted to keep the container lean.

After installation, a known library conflict is resolved by removing \texttt{libhsa-runtime64.so*} from the \texttt{torch} package’s \texttt{lib} directory. This library is bundled with some builds of \texttt{torch}, but its presence can interfere with the ROCm runtime already installed via the meta-package. A short inline Python script uses the \texttt{pathlib} and \texttt{glob} modules to locate and remove this file:

\begin{quote}
	\texttt{for so in glob.glob(str(libdir / 'libhsa-runtime64.so*')): os.remove(so)}
\end{quote}

This proactive cleanup step ensures that the final container relies on a single, consistent copy of the ROCm runtime, avoiding ambiguous or conflicting shared library references at runtime.

\vspace{1em}

\begin{minipage}{\textwidth}
	\captionsetup{type=listing}
	\caption*{Listing~\ref{lst:dockerfile}~(continued)}

	\begin{minted}[firstnumber=52,
                    breaklines,
                    breakanywhere=true,
                    fontsize=\footnotesize]{docker}
                    # ───────────────────────── 3. project requirements ───────────────────────────
                    WORKDIR /workspace
                    COPY requirements.txt .
                    RUN pip install --no-cache-dir -r requirements.txt

    \end{minted}
\end{minipage}

This block installs project-specific Python dependencies by referencing the \texttt{requirements.txt} file, which is first copied into the container’s working directory \texttt{/workspace}. The file contains only two entries: \texttt{sentencepiece} and \texttt{protobuf}.

The \texttt{sentencepiece} package is a widely used tokenizer library, particularly important for transformer-based language models trained with subword units. The \texttt{protobuf} package provides Python bindings for Protocol Buffers, a language-neutral serialization format commonly used in machine learning pipelines for model configuration and data exchange.

Dependencies are installed using \texttt{pip install --no-cache-dir -r requirements.txt}, which ensures a clean installation without storing wheel files in a local cache. This practice reduces the final container image size and avoids unnecessary clutter from temporary files.

Placing this block after all system-level configuration steps (e.g., ROCm stack and PyTorch installation) ensures a clear separation between infrastructure dependencies and lightweight application-level requirements. This ordering also improves reproducibility and simplifies debugging if Python package-related issues arise.

\vspace{1em}

\begin{minipage}{\textwidth}
	\captionsetup{type=listing}
	\caption*{Listing~\ref{lst:dockerfile}~(continued)}

	\begin{minted}[firstnumber=57,
                    breaklines,
                    breakanywhere=true,
                    fontsize=\footnotesize]{docker}
                    # ───────────────────────── 4. BigCode evaluation harness ─────────────────────
                    # Shallow clone (depth-1) and editable install *without* re-installing common deps
                    RUN git clone --depth 1 https://github.com/bigcode-project/bigcode-evaluation-harness.git \
                            /workspace/bigcode-evaluation-harness && \
                        pip install --no-cache-dir --no-deps -e /workspace/bigcode-evaluation-harness
                   
                    # Copy and enable the entrypoint
                    COPY src/scripts/entrypoint.sh /usr/local/bin/entrypoint.sh
                    RUN chmod +x /usr/local/bin/entrypoint.sh
                   
                    ENTRYPOINT ["/usr/local/bin/entrypoint.sh"]
                   
                    WORKDIR /workspace
                    CMD ["bash"]

    \end{minted}
\end{minipage}

This final block integrates the \texttt{BigCode} evaluation harness into the container and sets up the runtime environment.

First, the repository \texttt{bigcode-evaluation-harness} is cloned using a shallow \texttt{git clone} with the \texttt{--depth 1} flag. This optimization fetches only the latest commit, significantly reducing both the download size and build time. The repository is cloned into \texttt{/workspace/bigcode-evaluation-harness} and installed in editable mode using \texttt{pip install -e}. The \texttt{--no-deps} flag ensures that no dependencies are (re)installed, avoiding conflicts or duplication with packages already installed globally or via \texttt{requirements.txt}. This approach allows the harness to be directly editable within the container, which is particularly useful for development and debugging.

Next, the script \texttt{entrypoint.sh} is copied to \texttt{/usr/local/bin/} and made executable using \texttt{chmod +x}. This script acts as the container's \texttt{ENTRYPOINT}, meaning it will be executed automatically when the container starts. The use of a custom entrypoint script is common practice for injecting runtime logic such as token authentication, last-minute dependency checks, or configuration of environment variables.

Finally, the working directory is set to \texttt{/workspace}, and the \texttt{CMD} instruction specifies that the container should launch into a \texttt{bash} shell by default if no other command is provided. This setup allows for interactive use of the container during development, debugging, or manual inspection.

\vspace{1em}

\subsection{Entrypoint}

\begin{minipage}{\textwidth}
	\captionsetup{type=listing}
	\caption{Custom entrypoint script for runtime dependency management and non-interactive authentication}
	\label{lst:entrypoint}

	\begin{minted}[linenos,
        breaklines,
        breakanywhere=true,
        fontsize=\footnotesize]{bash}
        #!/usr/bin/env bash
        set -e

        # If HF_HUB_TOKEN is set, log in non-interactively
        if [ -n "$HF_HUB_TOKEN" ]; then
        echo "Logging into Hugging Face CLI..."
        echo "$HF_HUB_TOKEN" | huggingface-cli login --stdin
        fi

        # 1. Install any missing pip packages
        for pkg in ninja einops transformers packaging setuptools wheel triton; do
        if ! pip show "$pkg" >/dev/null 2>&1; then
            echo "Installing $pkg..."
            pip install "$pkg"
        fi
        done

        # 2. Clone & install causal-conv1d if absent
        if [ ! -d /tmp/causal-conv1d ]; then
        git clone https://github.com/Dao-AILab/causal-conv1d /tmp/causal-conv1d
        pushd /tmp/causal-conv1d
            pip install .
        popd
        fi

        # 3. Clone & install mamba if absent
        if [ ! -d /tmp/mamba ]; then
        git clone https://github.com/state-spaces/mamba /tmp/mamba
        pushd /tmp/mamba
            pip install . --no-build-isolation
        popd
        fi

        # 4. Clean up clones
        rm -rf /tmp/causal-conv1d /tmp/mamba

        exec "$@"
    \end{minted}
\end{minipage}


The custom \texttt{entrypoint.sh} script is executed at container runtime and plays a critical role in completing the environment setup for ROCm-based model evaluation. Rather than bundling all dependencies during the Docker build phase, this script defers the installation of several key Python packages to runtime. This design decision was informed by persistent compatibility issues encountered during build-time when installing packages such as \texttt{transformers}, \texttt{triton}, and particularly \texttt{mamba} in conjunction with ROCm-specific builds of \texttt{PyTorch}.

The script first performs a non-interactive login to Hugging Face if the environment variable \texttt{HF\_HUB\_TOKEN} is set. This enables seamless access to gated models during benchmarking without requiring user input.

Next, it ensures that essential Python packages required by some models (e.g., \texttt{Codestral-7B}) are available. It checks whether each package is already installed using \texttt{pip show} and installs any missing ones via \texttt{pip install}. This includes utilities such as \texttt{einops}, \texttt{ninja}, \texttt{transformers}, and \texttt{triton}, which were excluded from the Dockerfile due to build-time conflicts that do not manifest at runtime.

A critical section of the script addresses the installation of two auxiliary repositories: \texttt{causal-conv1d} and \texttt{mamba}. These libraries are required for evaluation but must be built from source. Importantly, \texttt{mamba} fails to build correctly under ROCm when installed with default \texttt{pip} isolation. This is due to its \texttt{pyproject.toml} requesting CUDA-specific builds of \texttt{torch} during isolated setup, which results in misidentification of the available backend and failure of HIP detection logic in its \texttt{setup.py}.

To circumvent this issue, the script installs \texttt{mamba} using \texttt{pip install . --no-build-isolation}, ensuring that the ROCm-specific \texttt{torch} installation already present in the environment is correctly reused. This approach is consistent with the workaround described in a public GitHub issue \autocite{mamba_rocm_conflict}, where the root cause was traced to a broken default environment isolation procedure under ROCm. In that issue, the developers suggest using \texttt{--no-build-isolation} and, optionally, applying AMD's ROCm patch \autocite{rocm_patch_405} to set correct GPU warp sizes.

Once installation is complete, the script cleans up temporary source directories to avoid polluting the container filesystem. It then calls \texttt{exec "\$@"} to forward control to the container's command entrypoint. This ensures that the script behaves transparently while guaranteeing a conflict-free runtime environment.

\vspace{1em}

\clearpage

\subsection{Docker Compose}

\begingroup
\captionsetup{type=listing}
\captionof{listing}{\texttt{docker-compose.yml} service definition
	for ROCm-enabled model benchmarking under WSL2}
\label{lst:docker-compose}

\par\medskip

\begin{minted}[linenos,
                 breaklines,
                 breakanywhere=true,
                 fontsize=\footnotesize]{yaml}
services:
  sandbox:
    container_name: sandbox
    build:
      context: .
      dockerfile: Dockerfile  
    image: sandbox:latest
    env_file: .env
    environment:
      PYTHONUNBUFFERED: "1"
      HF_ALLOW_CODE_EVAL: "1"
      HF_DATASETS_ALLOW_CODE_EXECUTION: "1"
      HF_DATASETS_TRUST_REMOTE_CODE: "1" 
      PYTHONPATH: "/app/src"
      TF_CPP_MIN_LOG_LEVEL: "3"
      TF_FORCE_GPU_ALLOW_GROWTH: "true"
        
      DEEPSEEK_CODER_MODEL_PATH:          "/app/cache/deepseek_coder/snapshots/2a050a4c59d687a85324d32e147517992117ed30"
      CODEGEMMA_MODEL_PATH:               "/app/cache/codegemma/snapshots/078cdc51070553d1636d645c9a238f3b0914459a"
      CODELLAMA_MODEL_PATH:               "/app/cache/codellama/snapshots/b97340ccba60c3ccaf91882ffe470a1bbc32e32c"
      CODESTRAL_MODEL_PATH:               "/app/cache/codestral/snapshots/88085f9cdfa832c3aca8a0315a4520cf7558c947"
      MISTRAL_MODEL_PATH:                 "/app/cache/mistral/snapshots/e0bc86c23ce5aae1db576c8cca6f06f1f73af2db"
      QWEN_MODEL_PATH:                    "/app/cache/qwen/snapshots/c03e6d358207e414f1eca0bb1891e29f1db0e242"
      STARCODER_MODEL_PATH:               "/app/cache/starcoder/snapshots/bb9afde76d7945da5745592525db122d4d729eb1"

    # ROCm-on-WSL privileges
    cap_add: [SYS_PTRACE]
    security_opt: [seccomp:unconfined]
    privileged: true
    devices:
      - /dev/dxg
      - /dev/dri
    group_add: [video]
    ipc: host
    shm_size: 8g

    volumes:
      - ./src:/app/src:ro
      - ./results:/app/results:rw
      - ./bigcode_cache:/app/bigcode_cache

      # Qwen 2.5-Coder-7B entire model cache (snapshots+blobs)
      - /mnt/c/Users/Elias/.cache/huggingface/hub/models--Qwen--Qwen2.5-Coder-7B-Instruct:/app/cache/qwen:ro
      # Meta CodeLlama 7B Python entire model cache (snapshots+blobs)
      - /mnt/c/Users/Elias/.cache/huggingface/hub/models--meta-llama--CodeLlama-7b-Python-hf:/app/cache/codellama:ro
      # BigCode StarCoder2 7B entire model cache (snapshots+blobs)
      - /mnt/c/Users/Elias/.cache/huggingface/hub/models--bigcode--starcoder2-7b:/app/cache/starcoder:ro
      # Mistral v0.3-7B Instruct entire model cache (snapshots+blobs) 
      - /mnt/c/Users/Elias/.cache/huggingface/hub/models--mistralai--Mistral-7B-Instruct-v0.3:/app/cache/mistral:ro
      # Mamba-Codestral-7B-v0.1 entire model cache (snapshots+blobs) 
      - /mnt/c/Users/Elias/.cache/huggingface/hub/models--mistralai--Mamba-Codestral-7B-v0.1:/app/cache/codestral:ro
      # Google Codegemma 7B it entire model cache (snapshots+blobs) 
      - /mnt/c/Users/Elias/.cache/huggingface/hub/models--google--codegemma-7b-it:/app/cache/codegemma:ro
      # Deepseek Coder 7B Instruct v1.5 entire model cache (snapshots+blobs) 
      - /mnt/c/Users/Elias/.cache/huggingface/hub/models--deepseek-ai--deepseek-coder-7b-instruct-v1.5:/app/cache/deepseek_coder:ro

      # ROCm WSL helpers
      - /usr/lib/wsl/lib/libdxcore.so:/usr/lib/libdxcore.so:ro
      - /opt/rocm/lib/libhsa-runtime64.so.1:/opt/rocm/lib/libhsa-runtime64.so.1:ro

    working_dir: /app      
    tty: true

networks:
  default:
    name: sandbox_network
  \end{minted}
\endgroup

The provided \texttt{docker-compose.yaml} configuration defines a single service named \texttt{sandbox}, which encapsulates the full runtime environment for benchmarking code generation models under ROCm on WSL2. The service builds from the previously defined \texttt{Dockerfile} and attaches a variety of privileges, devices, environment variables, and volume mounts tailored to GPU-accelerated, large-model evaluation.

The \texttt{build} directive specifies the local build context (\texttt{.}) and the \texttt{Dockerfile} to use. The image is tagged \texttt{sandbox:latest}, and secrets such as Hugging Face tokens are injected via the \texttt{env\_file} directive referencing a local \texttt{.env} file. To clarify what files and folders are available at build time, the structure of the working directory is shown below:

\begin{forest} pic dir tree
	[/
	[bigcode\_cache/]
	[results/
	[bigcode/
	[humaneval/]
	[humanevalplus/]
	[mbpp/]
    [mbppplus/]
	]
	]
	[src/
	[scripts/
	[entrypoint.sh]
	]
	]
	[venv/]
	[.env]
	[docker-compose.yaml]
	[Dockerfile]
	[requirements.txt]
	]
\end{forest}


The \texttt{src/} directory, mounted read-only to \texttt{/app/src}, contains scripts and configuration files needed for benchmarking, such as the \texttt{entrypoint.sh} and evaluation logic. The \texttt{results/} directory is mounted read-write into \texttt{/app/results}, serving as the destination for metrics and generations produced during evaluation. Outputs are further organized into benchmark-specific folders like \texttt{humaneval}, \texttt{mbpp}, and \texttt{humaneval\_plus}, categorized by model.

To ensure GPU access under WSL2 with ROCm, several kernel-level privileges and device interfaces must be explicitly granted:
\begin{itemize}
	\item \texttt{cap\_add: SYS\_PTRACE} enables debugging and profiling tools that require ptrace system calls.
	\item \texttt{security\_opt: seccomp:unconfined} disables seccomp filtering to allow ROCm’s dynamic libraries to execute necessary syscalls.
	\item \texttt{privileged: true} elevates container privileges to allow ROCm GPU backend functionality within WSL2.
	\item \texttt{devices} maps GPU access interfaces such as \texttt{/dev/dxg} and \texttt{/dev/dri}.
	\item \texttt{group\_add: video} ensures group-level permissions for GPU access match those of the host user session.
\end{itemize}

The \texttt{shm\_size} field increases the container’s shared memory to \texttt{8g}, helping prevent segmentation faults and out-of-memory errors during large-model inference.

Several volumes are mounted into the container:
\begin{itemize}
	\item \texttt{./src} $\rightarrow$ \texttt{/app/src} (read-only): mounts evaluation scripts and supporting code.
	\item \texttt{./results} $\rightarrow$ \texttt{/app/results} (read-write): stores generation and metric JSON files grouped by model and benchmark.
	\item Read-only model caches from Hugging Face are mounted under \texttt{/app/cache}, ensuring deterministic access to model snapshots across Qwen, CodeLlama, StarCoder2, Mistral, and Codestral.
	\item ROCm helper libraries (\texttt{libdxcore.so} and \texttt{libhsa-runtime64.so.1}) are mounted from host paths to maintain runtime compatibility.
	\item \texttt{./bigcode\_cache} provides persistent local caching for BigCode tools like \texttt{bigcode-evaluation-harness}.
\end{itemize}

The \texttt{environment} section defines runtime behavior:
\begin{itemize}
	\item \texttt{PYTHONUNBUFFERED=1} enables unbuffered logging to STDOUT.
	\item \texttt{HF\_ALLOW\_CODE\_EVAL=1} is required by Hugging Face transformers for secure execution of code evaluation.
	\item \texttt{PYTHONPATH=/app/src} ensures internal modules (e.g., \texttt{tasks.apps.yaml}) are discoverable.
	\item \texttt{HUGGINGFACE\_HUB\_TOKEN} is automatically passed from \texttt{.env}.
	\item Each \texttt{MODEL\_PATH} (e.g., \texttt{QWEN\_MODEL\_PATH}) maps to the relevant snapshot directory inside the mounted cache, which avoids unpredictable re-downloading or version mismatch.
	\item \texttt{ZENO\_API\_KEY} provides secure access to the Zeno platform for visualizing experiment metrics.
\end{itemize}

The working directory is set to \texttt{/app}, and \texttt{tty: true} enables interactive container sessions. Though a dedicated network \texttt{sandbox\_network} is defined, this single-service configuration does not require inter-container communication.

Altogether, this configuration captures a reproducible and portable ROCm-based development environment, enabling high-throughput model evaluation using cached resources and minimal host-side interference.

\chapter{Results}
\label{ch:results}

\section{Benchmarking Overview and Comparative Evaluation Strategy}

This chapter presents the outcomes of our benchmarking experiments conducted on a curated set of self-hosted 7B code generation models. The goal of this evaluation is to assess the practical performance of locally executable models across multiple code-oriented benchmarks and to contextualize these findings against publicly reported results for larger open-weight models and commercial cloud-based systems.

As described in Chapter~\ref{ch:proof-of-concept}, all models were evaluated under uniform local conditions using the \texttt{bigcode-evaluation-harness}, ensuring consistency in input formatting, sampling configuration, and pass@1 computation. In total, four benchmarks were employed: MBPP and HumanEval---representing functional correctness evaluations on standard Python programming tasks and algorithmic reasoning problems, respectively---and their extended variants, MBPP+ and HumanEval+, which consist of additional or modified problem sets but remain within the scope of zero-shot evaluation in our setup. All tasks were executed without providing any example completions or prompts beyond the task description, in accordance with our focus on isolated functional synthesis.

The results from our self-hosted evaluations are then placed in direct comparison with published scores as reported on the EvalPlus Leaderboard~\autocite{evalplus_leaderboard}, which aggregates reported pass@1 performance for a broad range of high-performing models. These include state-of-the-art cloud-hosted models such as OpenAI’s GPT-4, Google’s Gemini series, and the recently unveiled DeepSeek suite of models, as well as larger open-weight models---ranging between 13B and 70B parameters---that we were unable to run locally due to hardware constraints. While these models represent the current frontier in coding performance, they also bring well-documented trade-offs in terms of data privacy, cost, and deployment transparency, as previously outlined in Chapters~\ref{ch:inleiding} and~\ref{ch:stand-van-zaken}.

By structuring our analysis across the four aforementioned benchmarks, we aim to explore the extent to which specialized, instruction-tuned 7B models can serve as viable alternatives in secure, on-premises environments. This aligns directly with our central research question (see Section~\ref{rq:main}), and specifically addresses several of the formulated sub-questions—most notably the comparative performance of local and cloud-based systems (Sub-question~\ref{sq:performance}), the feasibility of deploying these models under hardware constraints (Sub-question~\ref{sq:deployment}), and the potential implications for secure development workflows (Sub-question~\ref{sq:secure-dev}). Each subsection in this chapter focuses on one benchmark and is structured to highlight absolute performance across local models, followed by comparative interpretation using published leaderboard results.

Together, these evaluations provide the empirical foundation for the conclusions developed in Chapter~\ref{ch:conclusie}, where we synthesize our findings in light of the organizational, technical, and ethical implications of adopting self-hosted AI coding assistants.

\subsection{Benchmark Performance of Local 7B Models}

% LOCAL

\begin{center}
	\includegraphics[width=1\textwidth]{/Users/csoregh/Documents/Thesis/Analyze/mbpp_mbpp_local.png}
	\captionof{figure}{MBPP Pass@1 scores for locally deployed 7B parameter models}
	\label{fig:mbpp-local}
\end{center}

\begin{center}
	\includegraphics[width=1\textwidth]{/Users/csoregh/Documents/Thesis/Analyze/mbpp_mbpp+_local.png}
	\captionof{figure}{MBPP+ Pass@1 scores for locally deployed 7B parameter models}
	\label{fig:mbppplus-local}
\end{center}

Across the MBPP and MBPP+ benchmarks (Figures~\ref{fig:mbpp-local} and~\ref{fig:mbppplus-local}), a strong stratification emerges among the locally deployed 7B models. Qwen2.5-Coder-7B-Instruct dominates the MBPP benchmark with a pass@1 of 73.8\%, clearly separating itself from the rest of the cohort. The next highest scorer, DeepSeek-Coder-7B-Instruct-v1.5, trails at 58.4\%, while Google-CodeGemma-7B-it follows at 50.4\%. This top-three cluster surpasses the 50\% threshold and represents models likely benefiting from strong alignment with real-world coding patterns found in MBPP.

In contrast, a second cluster of mid-performing models—including Meta-CodeLlama-7B-Python (45.6\%) and BigCode-StarCoder2-7B (45.4\%)—demonstrates more modest accuracy, suggesting limitations either in fine-tuning quality or architecture-task fit. Mistral-v0.3-7B-Instruct and Mamba-Codestral-7B-v0.1 perform worst among the group, scoring just 38.2\% and 36.4\% respectively. These results reaffirm that, even within the same parameter class and hardware constraints, there is wide variability in performance depending on design choices and training pipelines.

Upon transitioning to MBPP+, which rephrases and diversifies the prompts, most models improve. DeepSeek-Coder-7B-Instruct-v1.5 climbs to 64.0\%, a gain of +5.6\%, and Google-CodeGemma-7B-it similarly improves to 56.6\% (+6.2\%). Interestingly, models from the lower tier also benefit—Mistral rises by +5.7\%, and Mamba-Codestral gains +3.8\%, suggesting that these models may generalize better when task phrasing is broadened.

\begin{center}
	\includegraphics[width=1\textwidth]{/Users/csoregh/Documents/Thesis/Analyze/mbpp_mbpp_delta_local.png}
	\captionof{figure}{Δ Pass@1 between MBPP+ and MBPP for locally deployed 7B models}
	\label{fig:mbpp-delta-local}
\end{center}

However, the standout anomaly is Qwen2.5-Coder-7B-Instruct, which uniquely drops in performance, falling from 73.8\% to 70.1\% (−3.7\%)—the only negative delta among all models (Figure~\ref{fig:mbpp-delta-local}). This suggests that Qwen2.5’s architecture or fine-tuning is highly optimized for the phrasing and structural regularity of MBPP. The additional prompt diversity introduced in MBPP+ appears to destabilize its completions rather than enhance them, reinforcing the notion that Qwen2.5’s high baseline reflects strong pattern matching rather than in-context adaptability.

\begin{center}
	\includegraphics[width=1\textwidth]{/Users/csoregh/Documents/Thesis/Analyze/humaneval_humaneval_local.png}
	\captionof{figure}{HumanEval Pass@1 scores for locally deployed 7B parameter models}
	\label{fig:humaneval-local}
\end{center}

\begin{center}
	\includegraphics[width=1\textwidth]{/Users/csoregh/Documents/Thesis/Analyze/humaneval_humaneval+_local.png}
	\captionof{figure}{HumanEval+ Pass@1 scores for locally deployed 7B parameter models}
	\label{fig:humanevalplus-local}
\end{center}

A dramatic shift occurs when analyzing the HumanEval results. The most notable reversal is Qwen2.5-Coder-7B-Instruct: from being the clear leader on MBPP, it drops to 28.7\% on HumanEval, ranking near the bottom (Figure~\ref{fig:humaneval-local}). This sudden regression illustrates a severe lack of algorithmic reasoning capability—Qwen2.5 appears effective for routine programming but fails when abstraction and problem solving are required.

In contrast, DeepSeek-Coder-7B-Instruct-v1.5 exhibits remarkable consistency. Already strong on MBPP, it leads again on HumanEval with 69.5\% pass@1. Google-CodeGemma-7B-it retains its upper-middle position with 56.7\%, while mid-performers such as Meta-CodeLlama-7B-Python and StarCoder2-7B hover around the 36–41\% range. Mamba-Codestral underperforms across all tasks, bottoming out at 17.7\%.

\begin{center}
	\includegraphics[width=1\textwidth]{/Users/csoregh/Documents/Thesis/Analyze/humaneval_humaneval_delta_local.png}
	\captionof{figure}{Δ Pass@1 between HumanEval+ and HumanEval for locally deployed 7B models}
	\label{fig:humaneval-delta-local}
\end{center}

HumanEval+ introduces prompt variety in problem descriptions while preserving the underlying logic. This shift proves universally difficult: all models see a decline in pass@1, as shown in Figure~\ref{fig:humaneval-delta-local}. DeepSeek-Coder-7B-Instruct-v1.5 drops −3.0\%, Google-CodeGemma −4.3\%, and Mistral −4.9\%, indicating that their instruction-following capabilities are not as robust when problem statements deviate from familiar templates. StarCoder2-7B and Meta-CodeLlama also decline by −4.3\% and −3.7\% respectively.

Interestingly, Qwen2.5 shows the smallest relative decline (−1.8\%)—but this is misleading. Rather than indicating robustness, this minimal delta likely reflects a floor effect: its already-low performance on HumanEval (28.7\%) leaves little room for further deterioration. It does not recover any ground on HumanEval+, finishing at just 26.8\%.

\subsubsection*{Interpretation of Δ Patterns Across Tasks}

The Δ patterns between base and + variants of both benchmarks offer critical insight into task suitability and model generalizability. On MBPP+, nearly all models improve. This implies that when tasks involve standard library functions, control flow, or data structures, models benefit from prompt variety and may have latent few-shot capabilities—even without explicit few-shot context. On the other hand, the across-the-board decline on HumanEval+ reveals that prompt diversity is not helpful when tasks require logical planning or multi-step reasoning. In these cases, base model capability—not prompt phrasing—is the limiting factor.

These trends directly relate to Sub-question~\ref{sq:performance} and Sub-question~\ref{sq:deployment}. From a performance perspective, DeepSeek and CodeGemma exhibit both algorithmic competence and adaptability across formats. Qwen2.5, by contrast, excels only in pattern-rich, low-complexity tasks and appears fragile under prompt perturbation or reasoning demands. From a deployment lens, this suggests that developers relying on Qwen2.5 may achieve great performance in boilerplate-heavy workflows but will need to fall back on alternative tools for debugging or novel algorithm design.

\subsubsection*{Cross-Benchmark Patterns and Implications}

Viewed together, the MBPP and HumanEval benchmark sets expose divergent strengths among 7B models:

\textbf{Task Specialization:} Qwen2.5-Coder-7B-Instruct is a specialist in functional programming tasks that require syntax correctness and API familiarity. Its failure to generalize to HumanEval points to a lack of deeper semantic or logical processing. In contrast, DeepSeek-Coder-7B-Instruct-v1.5 stands out as a generalist, consistently achieving top scores across both practical and algorithmic tasks.

\textbf{Instruction Tuning and Pretraining Quality:} The relative strength of CodeGemma, DeepSeek, and Meta-CodeLlama may be attributable to their pretraining on diverse, high-quality coding corpora and instruction datasets. These models display resilience to prompt variation and maintain performance under task pressure. Mamba-Codestral and Mistral, on the other hand, appear less robust—potentially due to narrower fine-tuning, lack of high-quality instructions, or incompatible model internals.

In summary, these local benchmarking results provide a performance map that directly addresses Sub-question~\ref{sq:performance} regarding model capability and Sub-question~\ref{sq:deployment} concerning hardware-efficient inference. They also lay the groundwork for upcoming comparisons with larger open-weight and proprietary cloud-based systems, which will further illuminate how close 7B models have come to closing the performance gap—especially under the constraints of local, secure deployment.

% LARGE OPEN-WEIGHT

\subsection{Benchmark Performance of Large Open-Weight Models}

\begin{center}
	\includegraphics[width=1\textwidth]{/Users/csoregh/Documents/Thesis/Analyze/mbpp_mbpp_openweight.png}
	\captionof{figure}{MBPP Pass@1 scores for top open-weight models}
	\label{fig:mbpp-openweight}
\end{center}

\begin{center}
	\includegraphics[width=1\textwidth]{/Users/csoregh/Documents/Thesis/Analyze/mbpp_mbpp+_openweight.png}
	\captionof{figure}{MBPP+ Pass@1 scores for top open-weight models}
	\label{fig:mbppplus-openweight}
\end{center}

Across the MBPP and MBPP+ benchmarks (Figures~\ref{fig:mbpp-openweight} and~\ref{fig:mbppplus-openweight}), large open-weight models display a dense clustering at the top end of the pass@1 spectrum. OpenCodeInterpreter-DS-33B leads with 79.3\%, followed closely by speechless-codellama-34B-v2.0 (77.4\%) and starchat2-15b-v0.1 (73.8\%). All three exceed the 70\% threshold, signaling state-of-the-art performance on functional Python programming tasks.

Mid-range models like Code-13B, Code-33B, and OpenHermes-2.5-Code-290k-13B fall into the 54--56\% range, forming a second tier with relatively tight performance spread. At the lower end, StarCoder2-15B (46.3\%) and SOLAR-10.7B-Instruct-v1.0 (43.3\%) round out the leaderboard. These models may suffer from either insufficient instruction tuning or lack of architectural specialization for coding.

\begin{center}
	\includegraphics[width=1\textwidth]{/Users/csoregh/Documents/Thesis/Analyze/mbpp_mbpp_delta_openweight.png}
	\captionof{figure}{\(\Delta\) Pass@1 between MBPP+ and MBPP for top open-weight models}
	\label{fig:mbpp-delta-openweight}
\end{center}

Most models show a performance decline on MBPP+ (Figure~\ref{fig:mbpp-delta-openweight}), contrary to the trend observed in our local 7B benchmarks. StarCoder2-15B is the most affected, with a steep drop of −8.5\%, followed by starcoder2-15b-instruct-v0.1 at −7.3\%. Larger models like OpenCodeInterpreter-DS-33B, speechless-codellama-34B-v2.0, and Code-33B all lose −5.4--5.5\%, suggesting that even models with massive parameter counts are not immune to the destabilizing effects of prompt variation.

starchat2-15b-v0.1 experiences the smallest drop (−2.5\%), which may reflect better alignment with diverse prompts or superior instruction-following capabilities. However, no model improves, reinforcing that MBPP+—despite being syntactically similar to MBPP—can impose real challenges when prompts deviate from canonical phrasing.

\begin{center}
	\includegraphics[width=1\textwidth]{/Users/csoregh/Documents/Thesis/Analyze/humaneval_humaneval_openweight.png}
	\captionof{figure}{HumanEval Pass@1 scores for top open-weight models}
	\label{fig:humaneval-openweight}
\end{center}

\begin{center}
	\includegraphics[width=1\textwidth]{/Users/csoregh/Documents/Thesis/Analyze/humaneval_humaneval+_openweight.png}
	\captionof{figure}{HumanEval+ Pass@1 scores for top open-weight models}
	\label{fig:humanevalplus-openweight}
\end{center}

The HumanEval results (Figures~\ref{fig:humaneval-openweight} and~\ref{fig:humanevalplus-openweight}) mirror the MBPP trends. OpenCodeInterpreter-DS-33B again tops the chart at 79.3\%, with speechless-codellama-34B-v2.0 (77.4\%) and starchat2-15b-v0.1 (73.8\%) close behind. These models consistently demonstrate algorithmic competence across both base and + variants.

Lower-performing entries, such as SOLAR-10.7B-Instruct-v1.0 (43.3\%) and StarCoder2-15B (46.3\%), reinforce that scale alone does not guarantee robustness in structured problem-solving.

\begin{center}
	\includegraphics[width=1\textwidth]{/Users/csoregh/Documents/Thesis/Analyze/humaneval_humaneval_delta_openweight.png}
	\captionof{figure}{\(\Delta\) Pass@1 between HumanEval+ and HumanEval for top open-weight models}
	\label{fig:humaneval-delta-openweight}
\end{center}

HumanEval+ introduces more natural or diverse problem descriptions while preserving the underlying logic. Every model suffers a performance decline, with StarCoder2-15B showing the worst drop at −8.5\%. starcoder2-15b-instruct-v0.1 and speechless-starcoder2-15b also fall sharply (−7.3\% and −4.3\% respectively). Even leaders like OpenCodeInterpreter-DS-33B and Code-33B decline by −5.5\%, pointing to the inherent challenge of generalizing to varied prompt phrasing.

\subsubsection*{Cross-Benchmark Synthesis}

When viewed holistically, these large open-weight models show high peak performance but comparatively brittle generalization. Unlike locally deployed 7B models—many of which improved on MBPP+—larger models often regress under prompt variation. This contrast suggests that small models may benefit from tighter instruction tuning, while larger models may overfit to training distributions or exhibit fragility at scale.

For Sub-question~\ref{sq:performance}, these results confirm that although open-weight giants can match cloud-grade accuracy on curated tasks, their behavior under real-world conditions may falter. Subtle prompt changes lead to double-digit percentage losses in some cases. From a deployment standpoint (Sub-question~\ref{sq:deployment}), this inconsistency demands caution: deploying large open-weight models may require heavier prompt engineering or downstream validation pipelines to ensure safety and reliability.

This comparative section establishes an upper-bound reference for self-hosted performance. The subsequent cloud model analysis will extend this contrast further, offering insight into whether proprietary models outperform open-weight systems not just in average score, but also in adaptability and resilience under varied task settings.

% CLOUD

\subsection{Benchmark Performance of Cloud-Based Models}

\begin{center}
	\includegraphics[width=1\textwidth]{/Users/csoregh/Documents/Thesis/Analyze/mbpp_mbpp_cloud.png}
	\captionof{figure}{MBPP Pass@1 scores for top proprietary cloud-based models}
	\label{fig:mbpp-cloud}
\end{center}

\begin{center}
	\includegraphics[width=1\textwidth]{/Users/csoregh/Documents/Thesis/Analyze/mbpp_mbpp+_cloud.png}
	\captionof{figure}{MBPP+ Pass@1 scores for top proprietary cloud-based models}
	\label{fig:mbppplus-cloud}
\end{center}

On the MBPP benchmark (Figure~\ref{fig:mbpp-cloud}), cloud-based models dominate the performance frontier. o1 Preview (Sept 2024) and o1 Mini (Sept 2024) share the highest pass@1 score of 96.3\%, closely followed by GPT-4o (Aug 2024) and Qwen2.5-Coder-32B-Instruct, each with 92.7\% and 92.1\%, respectively. DeepSeek variants (V3, V2.5, and Coder-V2-Instruct) and GPT-4 models (April 2024, Nov 2023) round out the list with high-80\% scores, reinforcing that the proprietary frontier in code generation is advancing rapidly.

The MBPP+ results (Figure~\ref{fig:mbppplus-cloud}) confirm this dominance. All models remain above 81.7\% in pass@1 accuracy, with top entries like o1 Preview and o1 Mini holding steady at 89.0\%. These scores suggest strong instruction-following robustness and generalization to prompt rephrasings.

\begin{center}
	\includegraphics[width=1\textwidth]{/Users/csoregh/Documents/Thesis/Analyze/mbpp_mbpp_delta_cloud.png}
	\captionof{figure}{\(\Delta\) Pass@1 between MBPP+ and MBPP for top proprietary cloud models}
	\label{fig:mbpp-delta-cloud}
\end{center}

Despite their high absolute scores, nearly all models exhibit a negative delta on MBPP+ (Figure~\ref{fig:mbpp-delta-cloud}). Gemini 1.5 Pro 002 and GPT-4 (May 2023) decline the most, losing −9.7\% and −9.1\%, respectively. Even top models like o1 Preview and o1 Mini lose −7.3\%, indicating that prompt sensitivity remains an issue at the cutting edge.

GPT-4o, Qwen2.5-32B, and DeepSeek-V3 are among the least affected (−4.9\% each), highlighting relative stability but also implying potential gains from improved prompt robustness mechanisms.

\begin{center}
	\includegraphics[width=1\textwidth]{/Users/csoregh/Documents/Thesis/Analyze/humaneval_humaneval_cloud.png}
	\captionof{figure}{HumanEval Pass@1 scores for top proprietary cloud-based models}
	\label{fig:humaneval-cloud}
\end{center}

\begin{center}
	\includegraphics[width=1\textwidth]{/Users/csoregh/Documents/Thesis/Analyze/humaneval_humaneval+_cloud.png}
	\captionof{figure}{HumanEval+ Pass@1 scores for top proprietary cloud-based models}
	\label{fig:humanevalplus-cloud}
\end{center}

On HumanEval (Figure~\ref{fig:humaneval-cloud}), the pattern is similar. o1 Preview and o1 Mini again lead with 96.3\%, while GPT-4o, Qwen2.5-Coder-32B-Instruct, and DeepSeek-V3 all exceed 91\%. GPT-4 models from earlier releases still perform well, with pass@1 around 88.4\%--90.2\%.

HumanEval+ (Figure~\ref{fig:humanevalplus-cloud}) introduces rephrased problems and natural language variance, and once again all models register performance drops.

\begin{center}
	\includegraphics[width=1\textwidth]{/Users/csoregh/Documents/Thesis/Analyze/humaneval_humaneval_delta_cloud.png}
	\captionof{figure}{\(\Delta\) Pass@1 between HumanEval+ and HumanEval for top proprietary cloud models}
	\label{fig:humaneval-delta-cloud}
\end{center}

As shown in Figure~\ref{fig:humaneval-delta-cloud}, Gemini 1.5 Pro 002 and GPT-4 (May 2023) again incur the steepest penalties, dropping −9.7\% and −9.1\%, respectively. o1 Preview, o1 Mini, and GPT-4o all drop by −7.3\% to −5.5\%. Only GPT-4-Turbo (April 2024) exhibits a relatively modest decline (−3.6\%), reflecting a rare case of prompt resilience.

\subsubsection*{Cloud Performance Summary and Implications}

These results demonstrate that proprietary cloud models consistently achieve the highest pass@1 scores across all benchmarks. However, they are not immune to prompt variability. All top performers decline on both MBPP+ and HumanEval+, with losses typically in the 5--10\% range.

This recurring sensitivity suggests that even best-in-class systems depend heavily on prompt form and phrasing. While this fragility originates from a much higher baseline compared to open-weight and local models, it reinforces the need to evaluate not just raw capability, but also consistency under prompt variation.

The next chapter (Chapter~\ref{ch:conclusie}) will provide final answers to the research question and supporting sub-questions based on the full body of benchmark results.


%%=============================================================================
%% Conclusie
%%=============================================================================

\chapter{\IfLanguageName{dutch}{Conclusie}{Discussion}}%
\label{ch:conclusie}

% TODO: Trek een duidelijke conclusie, in de vorm van een antwoord op de
% onderzoeksvra(a)g(en). Wat was jouw bijdrage aan het onderzoeksdomein en
% hoe biedt dit meerwaarde aan het vakgebied/doelgroep? 
% Reflecteer kritisch over het resultaat. In Engelse teksten wordt deze sectie
% ``Discussion'' genoemd. Had je deze uitkomst verwacht? Zijn er zaken die nog
% niet duidelijk zijn?
% Heeft het onderzoek geleid tot nieuwe vragen die uitnodigen tot verder 
%onderzoek?

This thesis set out to evaluate whether self-hosted \glspl{LLM} offer a viable, secure, and performant alternative to proprietary cloud-based systems for code generation. Using a reproducible evaluation pipeline (Chapter~\ref{ch:methodologie}), we benchmarked eight open-weight 7B models locally under consumer-grade GPU constraints and compared their performance with top-tier open-weight and proprietary systems. Functional correctness was assessed using four well-established benchmarks—MBPP, MBPP+, HumanEval, and HumanEval+—selected to represent both practical software tasks and algorithmic reasoning challenges.

\section{Modern self-hosted LLMs can serve as viable, secure alternatives to proprietary cloud-based AI models for code generation}

To address the main research question—\textit{"Can modern self-hosted LLMs serve as viable, secure alternatives to proprietary cloud-based AI models for code generation?"}~(\ref{rq:main})—we synthesized results across all benchmarks and model classes (Chapter~\ref{ch:results}). Based on our findings, the answer is conditionally affirmative. The performance gap between larger open-weight models (13–34 B parameters) and the top proprietary cloud services is surprisingly narrow. On MBPP, several 13–33 B open-weight systems achieve pass@1 scores in the high seventies to low eighties (Figure~\ref{fig:mbpp-openweight}), trailing the best cloud offerings by only 10–15 points (Figure~\ref{fig:mbpp-cloud}). Likewise, on HumanEval, these larger open models score in the 70–75\% range, compared to roughly 90\% for GPT-4 and Gemini Ultra—again a gap of 15–20 points (Figure~\ref{fig:humaneval-cloud}). In practical terms, this means that teams who choose cutting-edge open-weight checkpoints can already deliver code-generation quality that approaches, and almost matches, the cloud frontier.

More notably, our locally hosted 7 B parameter models close much of that remaining distance. Qwen2.5-Coder-7B-Instruct hits 73.8\% on MBPP (Figure~\ref{fig:mbpp-local}), putting it within 5 points of those hefty 13–33 B open-weight peers—and just 20 points behind commercial services. DeepSeek-Coder-7B-Instruct reaches 69.5\% on HumanEval (Figure~\ref{fig:humaneval-local}), outperforming many larger public models. Crucially, both run entirely on a single AMD RX 7900 XT with 20 GB VRAM, using 4-bit quantization and a reproducible Docker+ROCm setup. This demonstrates that organizations of any size—whether a two-person startup or a mid-sized enterprise—can deploy competitive code assistants without investing in massive cloud commitments or specialized hardware.

All models, cloud and local, still exhibit sensitivity to prompt design (see the Δ plots in Figures~\ref{fig:mbpp-delta-local} and~\ref{fig:humaneval-delta-cloud}), underscoring that robustness and consistency must complement raw pass@1 accuracy in any production scenario. Nevertheless, our findings confirm that the luxury of near–state-of-the-art code generation is no longer limited to organizations with large budgets or access to datacenter-scale GPUs.

\section{Research sub-question Review}

\paragraph{SQ1: Which self-hosted LLMs offer the best support for coding tasks while minimizing data leakage risks?}
DeepSeek-Coder-7B-Instruct emerged as the most consistent performer among self-hosted models. It ranked highest on HumanEval (69.5\%) and held strong across HumanEval+ and MBPP+ (Figures~\ref{fig:humanevalplus-local} and~\ref{fig:mbppplus-local}), indicating resilience to prompt variation and aptitude for both algorithmic and real-world tasks. Qwen2.5-Coder-7B-Instruct outperformed all peers on MBPP (73.8\%) but dropped sharply on HumanEval (28.7\%, Figure~\ref{fig:humaneval-local}), revealing a specialization in pattern-based tasks rather than abstract reasoning. Since both models can run entirely offline, they present strong candidates for minimizing data exposure in enterprise or sensitive development contexts.

\paragraph{SQ2: Can self-hosted LLMs reliably assist in secure software development without introducing vulnerabilities?}
When used within a human-in-the-loop setting, our benchmarks indicate that top-performing local models provide functionally correct outputs at a rate sufficient to support development workflows. While some completions contain logic errors or misunderstood instructions (especially in HumanEval+), no catastrophic failures or security-compromising behaviors were observed under test conditions. Local execution under Docker, ROCm, and WSL2 (Section~\ref{sec:pipeline}) allows for air-gapped deployment and deterministic inference, supporting high-integrity use cases with minimized attack surfaces.

\paragraph{SQ3: How does the performance of self-hosted LLMs compare to cloud-based AI tools in coding tasks?}
The strongest proprietary systems—including GPT-4o and o1 Preview—achieve pass@1 scores upwards of 96\% on both MBPP and HumanEval (Figures~\ref{fig:mbpp-cloud} and~\ref{fig:humaneval-cloud}). Nonetheless, the best local 7B models now operate within 15--20 percentage points of these models across benchmarks, with considerably narrower margins in tasks such as MBPP (Figure~\ref{fig:mbppplus-local}). The performance delta is no longer prohibitive, particularly for organizations prioritizing privacy, reproducibility, or infrastructure control. Importantly, even cloud models experienced performance drops of 5--10\% on MBPP+ and HumanEval+ (Figures~\ref{fig:mbpp-delta-cloud} and~\ref{fig:humaneval-delta-cloud}), underscoring that robustness remains an open challenge across the board.

\paragraph{SQ4: What are the deployment challenges and resource requirements of running LLMs locally?}
All local benchmarks were executed on a single AMD RX 7900 XT GPU with 20GB VRAM, using 4-bit quantized weights and containerized orchestration. Inference latency ranged from one to 43 seconds depending on sequence length and prompt complexity. These findings confirm that 7B models are deployable under modest infrastructure constraints. Larger models—such as 13B or 34B open-weight variants—deliver stronger results but remain inaccessible to most local setups due to elevated VRAM and compute demands (Figure~\ref{fig:mbpp-openweight}).

\paragraph{SQ5: What best practices should organizations follow when integrating self-hosted AI coding assistants?}
Based on our evaluation, organizations should select models with strong benchmark performance (e.g., DeepSeek or Qwen2.5), permissive licensing, instruction tuning, and quantization support. Reproducible Docker containers with pinned versions, deterministic inference settings, and thorough logging are essential. Evaluation should extend beyond HumanEval to include MBPP, MBPP+, and task-specific probes (Chapter~\ref{ch:results}) to avoid overfitting to narrow capabilities. Lastly, deployment should always preserve human oversight and auditability, particularly in regulated or safety-critical environments.

\section{Final Reflections}

The rapid advance of open-weight LLMs now makes secure, on-premise code generation a practical reality. While proprietary models still lead in absolute performance, their margin of superiority is shrinking—and not always accompanied by better generalization. In some cases, locally deployed 7B models demonstrate greater prompt stability than their larger, cloud-based counterparts when transitioning from base to augmented benchmarks (Figures~\ref{fig:mbpp-delta-local},~\ref{fig:humaneval-delta-openweight},~\ref{fig:mbpp-delta-cloud}).

Perhaps most notably, the ability to deploy performant models locally no longer presupposes access to large-scale infrastructure. Our findings show that meaningful levels of functional accuracy are attainable using consumer-grade hardware and open-source tooling, effectively lowering the barrier to entry. This shifts the conversation from enterprise feasibility to strategic readiness: even smaller organizations now possess the technical and financial means to operate secure AI systems in-house.

This thesis has demonstrated that self-hosted LLMs, when carefully selected and deployed, can support real-world coding tasks with surprising accuracy, security, and flexibility. The challenge no longer lies in access to tooling—it lies in the precision of implementation. As future work, we recommend research into low-rank adaptation (LoRA) for task-specific fine-tuning, hybrid deployment models, and safety-focused evaluation protocols. What was once a theoretical alternative to cloud-based code generation has now matured into an actionable option.


%---------- Bijlagen -----------------------------------------------------------

\appendix

\chapter{Onderzoeksvoorstel}

Het onderwerp van deze bachelorproef is gebaseerd op een onderzoeksvoorstel dat vooraf werd beoordeeld door de promotor. Dat voorstel is opgenomen in deze bijlage.

%% TODO: 
\section*{Samenvatting}

Het voorstel richt zich op het benchmarken van zelf-gehoste \gls{LLM} om de effectiviteit
van deze modellen te evalueren in het veilig ondersteunen van programmeerprocessen binnen bedrijfsomgevin-
gen. Organisaties die gevoelige code verwerken, vermijden steeds vaker cloudgebaseerde \glsentryshort{AI}-modellen vanwege
het risico dat modellen zoals ChatGPT of Gemini deze code opnemen in hun trainingssets of opslaan, wat leidt
tot mogelijke blootstelling van bedrijfsgevoelige informatie. De onderzoeksvraag luidt: ”Welke zelf-gehoste LLM
biedt de beste ondersteuning voor codeertaken terwijl datalekrisico’s worden geminimaliseerd?”. Het doel van
de bachelorproef is om een gestructureerde evaluatie te bieden van de prestaties, nauwkeurigheid, efficiëntie
en gebruiksvriendelijkheid van geselecteerde open-source \glspl{LLM}, zoals StarCoder en Mistral. De methodologie
omvat het opstellen van gestandaardiseerde benchmarks, zoals HumanEval en \glsentryshort{MBPP}, waarmee de modellen
worden getest op een reeks programmeertaken in een gecontroleerde on-premise omgeving. Verwacht wordt
dat dit onderzoek niet alleen inzicht geeft in welke modellen technisch het meest geschikt zijn, maar ook prakti-
sche aanbevelingen biedt aan bedrijven voor het integreren van veilige \glsentryshort{AI}-oplossingen in hun ontwikkelproces-
sen. Hierdoor worden bedrijven ondersteund in hun zoektocht naar privacybewuste \glsentryshort{AI}-oplossingen, waarmee
innovatie kan worden bevorderd zonder in te boeten op de veiligheid van gevoelige gegevens.

% Kopieer en plak hier de samenvatting (abstract) van je onderzoeksvoorstel.

% Verwijzing naar het bestand met de inhoud van het onderzoeksvoorstel
%---------- Inleiding ---------------------------------------------------------

% TODO: Is dit voorstel gebaseerd op een paper van Research Methods die je
% vorig jaar hebt ingediend? Heb je daarbij eventueel samengewerkt met een
% andere student?
% Zo ja, haal dan de tekst hieronder uit commentaar en pas aan.

%\paragraph{Opmerking}

% Dit voorstel is gebaseerd op het onderzoeksvoorstel dat werd geschreven in het
% kader van het vak Research Methods dat ik (vorig/dit) academiejaar heb
% uitgewerkt (met medesturent VOORNAAM NAAM als mede-auteur).
% 

\section{Inleiding}%
\label{sec:inleiding}

Het thema van deze bachelorproef is het vergelijken van verschillende self-hosted Large Language Models (LLMs) om hun geschiktheid voor het ondersteunen van codeertaken binnen bedrijfsomgevingen te bepalen. Met de toenemende afhankelijkheid van bedrijven van AI voor programmeerhulp groeit ook de bezorgdheid over de veiligheid van gevoelige gegevens. Cloudgebaseerde modellen zoals ChatGPT of Gemini brengen het risico met zich mee dat bedrijfsgevoelige code wordt opgenomen in trainingssets, wat kan leiden tot datalekken. Voor bedrijven die data-privacy als prioriteit zien, bieden self-hosted LLMs een mogelijk veilig alternatief.

De onderzoeksvraag luidt:

"Welke self-hosted LLM biedt de beste ondersteuning voor codeertaken binnen bedrijven, terwijl de risico’s op datalekken worden geminimaliseerd?"

Om deze vraag te beantwoorden, worden de volgende deelvragen geformuleerd:

\begin{enumerate}
    \item Welke benchmarks zijn het meest geschikt om de prestaties van self-hosted LLMs te evalueren?

    \item Wat zijn de hardware- en softwarevereisten voor de implementatie van self-hosted LLMs?

    \item Hoe presteren de geselecteerde modellen op het gebied van nauwkeurigheid, responstijd en resourcegebruik?

    \item Hoe eenvoudig zijn self-hosted LLMs te integreren in bestaande ontwikkelworkflows?

\end{enumerate}

Het doel van dit onderzoek is om bedrijven een objectieve evaluatie te bieden van de prestaties, nauwkeurigheid, efficiëntie en gebruiksvriendelijkheid van geselecteerde open-source LLMs, zoals StarCoder en Mistral. Hierbij worden modellen getest met behulp van gestandaardiseerde benchmarks, zoals HumanEval en MBPP, in een gecontroleerde self-hosted omgeving.

%---------- Stand van zaken ---------------------------------------------------

\section{Literatuurstudie}%
\label{sec:literatuurstudie}

De snelle ontwikkeling van Large Language Models (LLMs) heeft geleid tot brede toepassingen in uiteenlopende domeinen, variërend van tekstgeneratie en vertaling tot programmeerassistentie. Traditioneel zijn deze modellen in cloudomgevingen gehost, zoals bij OpenAI's ChatGPT en het binnenkort beschikbare Gemini van Google. Deze cloudgebaseerde oplossingen bieden aanzienlijke voordelen op het gebied van schaalbaarheid, onderhoud en directe toegankelijkheid \autocite{Brown2020}. Toch groeit de vraag naar self-hosted alternatieven snel, vooral binnen bedrijven die waarde hechten aan dataprivacy en volledige controle over hun infrastructuur. Een belangrijk aandachtspunt hierbij is het risico dat bedrijfsgevoelige data – inclusief broncode en intellectueel eigendom – onbedoeld terechtkomt in externe trainingsdatasets of op systemen buiten de eigen controle \autocite{Rao2023}.

Self-hosted LLMs proberen deze uitdagingen aan te pakken door bedrijven in staat te stellen de modellen lokaal, of op hun eigen servers, te draaien. Hierdoor behouden organisaties de regie over de opslag en verwerking van hun data en minimaliseren zij het risico op datalekken. Modellen zoals StarCoder, ontwikkeld door Hugging Face, en Mistral, een compact maar krachtig open-source model, zijn specifiek ontworpen met deze doelstellingen voor ogen. StarCoder richt zich met name op codeertaken en ondersteunt meerdere programmeertalen, waardoor het een brede inzetbaarheid kent in softwareontwikkelingsteams \autocite{Li2022, Tunstall2023}. Mistral is daarentegen geoptimaliseerd voor efficiëntie en lagere hardwarevereisten, wat het aantrekkelijk maakt voor bedrijven met beperktere infrastructuur \autocite{Mistral2023, Team2023}.

Eerdere studies hebben al laten zien dat benchmarks zoals HumanEval en MBPP waardevol zijn om de codegeneratieprestaties van LLMs te meten \autocite{Chen2021}. Deze benchmarks testen niet alleen de nauwkeurigheid van de gegenereerde code, maar ook het probleemoplossend vermogen van de modellen in gestandaardiseerde scenario’s. Tot nu toe lag de focus daarbij voornamelijk op cloudgebaseerde oplossingen, zoals OpenAI Codex, die indrukwekkende resultaten hebben laten zien. Echter, het gebrek aan vergelijkend onderzoek naar self-hosted alternatieven en hun toepasbaarheid in een bedrijfsomgeving blijft een belangrijke leemte in de literatuur.

Deze leemte wordt nog duidelijker wanneer men de praktische aspecten van de implementatie van self-hosted modellen in beschouwing neemt. Hoewel er case studies zijn die bepaalde technische aspecten en resourcevereisten van self-hosted LLMs belichten \autocite{Rao2023}, blijft er een gebrek aan systematische evaluaties die niet alleen prestaties, maar ook integratiegemak, veiligheid en compatibiliteit met bestaande ontwikkelomgevingen onder de loep nemen. Dit zijn juist de factoren die in een bedrijfscontext essentieel zijn voor de adoptie en het succes van dergelijke technologie.

Dit onderzoek bouwt voort op bestaande literatuur, maar gaat verder door ook de volgende concrete vragen te adresseren: welke benchmarks zijn het meest geschikt om de prestaties van self-hosted LLMs te evalueren, welke hardware- en softwarevereisten zijn noodzakelijk, hoe presteren de gekozen modellen op het gebied van nauwkeurigheid, responstijd en resourcegebruik, en in hoeverre kunnen ze eenvoudig worden geïntegreerd in bestaande ontwikkelworkflows? Door deze aspecten samen te brengen, zal deze studie niet alleen inzicht bieden in de technische prestaties van modellen als StarCoder en Mistral, maar ook een praktische richtlijn vormen voor bedrijven die de overstap naar self-hosted LLMs overwegen. Daarmee draagt dit onderzoek bij aan een meer holistische benadering van LLM-evaluatie, waarin zowel technische als organisatorische factoren worden meegenomen.
%---------- Methodologie ------------------------------------------------------
\section{Methodologie}%
\label{sec:methodologie}

Dit onderzoek volgt een systematische aanpak om de onderzoeksvraag te beantwoorden. De onderzoeksmethoden zijn ingericht rond de eerder geformuleerde deelvragen:

\begin{enumerate}
    \item Welke benchmarks zijn het meest geschikt om de prestaties van self-hosted LLMs te evalueren?

    \item Wat zijn de hardware- en softwarevereisten voor de implementatie van self-hosted LLMs?

    \item Hoe presteren de geselecteerde modellen op het gebied van nauwkeurigheid, responstijd en resourcegebruik?

    \item Hoe eenvoudig zijn self-hosted LLMs te integreren in bestaande ontwikkelworkflows?

\end{enumerate}

Door het combineren van literatuurstudie, experimentele evaluaties en een Proof of Concept (PoC), wordt een geïntegreerd beeld verkregen. Onderstaande stappen en technieken zijn specifiek gekoppeld aan de deelvragen.

\subsection{Onderzoekstechnieken en stappen}

\begin{enumerate}
    \item \textbf{Literatuurstudie}
    \begin{itemize}
        \item In deze fase worden bestaande studies, documentatie en vakliteratuur onderzocht om inzicht te krijgen in welke benchmarks momenteel worden gebruikt voor het evalueren van codegerichte LLMs. Daarnaast worden de hardware- en softwarevereisten in kaart gebracht aan de hand van documentatie van modelontwikkelaars en ervaringen uit de praktijk. Dit leidt tot:
        \begin{itemize}
            \item Identificatie van relevante benchmarks zoals HumanEval en MBPP \autocite{Chen2021}.
            \item Het opstellen van een overzicht van noodzakelijke infrastructuur (bijvoorbeeld NVIDIA GPU’s, Docker, Kubernetes) voor het zelf hosten van LLMs.
        \end{itemize}
    \end{itemize}

    \item \textbf{Modelselectie en installatie}
    \begin{itemize}
        \item Op basis van de literatuurstudie worden 3-5 self-hosted LLMs geselecteerd (bijvoorbeeld StarCoder, Mistral). Deze modellen worden geïnstalleerd in een gecontroleerde testomgeving. Hierbij wordt gelet op:
        \begin{itemize}
            \item De vereisten voor hardware (GPU’s) en software (model-implementaties, container- en orkestratietools).
            \item Het controleren van de haalbaarheid en complexiteit van de set-up, wat inzichten biedt voor deelvraag 2.
        \end{itemize}
    \end{itemize}

    \item \textbf{Ontwikkeling van benchmarks en testcases}
    \begin{itemize}
        \item Op basis van de literatuurstudie worden benchmarks en testcases samengesteld om de prestaties van de geselecteerde modellen te meten. Hierbij gaat het om:
        \begin{itemize}
            \item Het selecteren of aanpassen van bestaande benchmarks (HumanEval, MBPP) voor codegeneratietaken, zodat zij representatief zijn voor bedrijfsrelevante scenario’s.
            \item Het opstellen van prestatie-indicatoren als nauwkeurigheid (correctheid van gegenereerde code), responstijd (tijd tot bruikbaar antwoord), en resourcegebruik (GPU-gebruik, geheugengebruik).
            Dit beantwoordt deels deelvraag 1 (keuze van benchmarks) en vormt de meetlat voor deelvraag 3 (prestatie-indicatoren).
        \end{itemize}
    \end{itemize}

    \item \textbf{Experimenten uitvoeren}
    \begin{itemize}
        \item Alle geselecteerde modellen worden onder identieke omstandigheden getest. Hierbij wordt:
        \begin{itemize}
            \item De nauwkeurigheid bepaald aan de hand van gestandaardiseerde testcases.
            \item De responstijd gemeten op identieke codeerproblemen.
            \item Het resourcegebruik (GPU, CPU, geheugen) bijgehouden.
            \item Ook worden veiligheidscontroles uitgevoerd (bijv. controle op datalekken of externe dataopslag) om de geschiktheid voor een bedrijfsomgeving te beoordelen.
            De uitkomsten van deze experimenten geven een duidelijk antwoord op deelvraag 3.
            Dit beantwoordt deels deelvraag 1 (keuze van benchmarks) en vormt de meetlat voor deelvraag 3 (prestatie-indicatoren).
        \end{itemize}
    \end{itemize}

    \item \textbf{Proof of Concept (PoC)}
    \begin{itemize}
        \item Om te onderzoeken hoe eenvoudig de modellen kunnen worden geïntegreerd in bestaande ontwikkelomgevingen, wordt een PoC opgezet. Hierbij wordt een geselecteerd model geïntegreerd in een ontwikkelworkflow:
        \begin{itemize}
            \item Inzet van een LLM in een lokaal geïnstalleerde IDE (bijvoorbeeld Visual Studio Code) met AI-plug-ins.
            \item Evaluatie van de gebruiksvriendelijkheid, de benodigde aanpassingen aan de ontwikkelworkflow en de daadwerkelijke tijdswinst en codekwaliteit.
            Dit proces levert direct inzicht in de praktische toepasbaarheid en beantwoordt deelvraag 4.
        \end{itemize}
    \end{itemize}
\end{enumerate}

\subsection{Tools en technologieën}

\begin{itemize}
    \item \textbf{Hardware:} Hardware: NVIDIA GPU’s (A100/H100) voor optimale prestatie.
    \item \textbf{Software:} Software: Docker, Kubernetes voor modelhosting en schaalbaarheid; Python voor scripts en analyses; benchmarking tools (bijvoorbeeld OpenAI’s evaluation scripts \autocite{Brown2020}).
    \item \textbf{Benchmarks:} Benchmarks: HumanEval, MBPP, mogelijk aangevuld met bedrijfsrelevante testcases.
\end{itemize}

\subsection{Tijdschatting en deliverables}

\begin{itemize}
    \item \textbf{Literatuurstudie en modelselectie} (3 weken)  
    Deliverable: Overzicht relevante benchmarks, infrastructuurvereisten, modelkeuze.

    \item \textbf{Modelimplementatie en testomgeving opzetten} (4 weken)  
    Deliverable: Werkende self-hosted implementaties van geselecteerde modellen.

    \item \textbf{Ontwikkeling en validatie van benchmarks} (2 weken)  
    Deliverable: Een set gestandaardiseerde testcases en een geconfigureerde testomgeving.

    \item \textbf{Uitvoeren van experimenten} (4 weken)  
    Deliverable: Data en grafieken met prestatie-indicatoren en vergelijkingen.

    \item \textbf{Proof of Concept} (2 weken)  
    Deliverable: Werkende integratie van een LLM in een lokale ontwikkelworkflow, inclusief documentatie en bevindingen.

    \item \textbf{Analyse en rapportage} (3 weken)  
    Deliverable: Eindrapport met conclusies, aanbevelingen en beantwoording van alle deelvragen.
\end{itemize}

Door deze aanpak worden alle deelvragen direct of indirect beantwoord en ontstaat een volledig beeld van de praktische inzetbaarheid van self-hosted LLMs in een bedrijfscontext.

%---------- Verwachte resultaten ----------------------------------------------
\section{Verwacht resultaat, conclusie}%
\label{sec:verwachte_resultaten}

\subsection{Verwacht resultaat}

Het onderzoek zal resulteren in een systematische evaluatie van self-hosted Large Language Models (LLMs) zoals StarCoder en Mistral op basis van hun prestaties en geschiktheid voor code-assistentie in bedrijfsomgevingen. De verwachte resultaten worden hierbij gerelateerd aan de gestelde deelvragen:
\begin{itemize}
    \item \textbf{(deelvraag 1)} Benchmarks
    \begin{itemize}
        \item Een duidelijk overzicht van welke benchmarks (bijvoorbeeld HumanEval, MBPP) het meest geschikt zijn voor het evalueren van self-hosted LLMs. Dit omvat een onderbouwing waarom deze benchmarks representatief zijn voor reële programmeerproblemen binnen bedrijven.
    \end{itemize}

    \item \textbf{(deelvraag 2)} Implementatievereisten
    \begin{itemize}
        \item Een specificatie van de benodigde hardware- en softwarevereisten voor het zelf hosten van LLMs, inclusief aanbevelingen voor GPU-configuraties, container- en orkestratietools, en richtlijnen voor het optimaal inzetten van modelresources.
    \end{itemize}

    \item \textbf{Praktische toepasbaarheid:} Een beoordeling van hoe eenvoudig deze modellen kunnen worden geïntegreerd in een typische ontwikkelworkflow, inclusief hun compatibiliteit met tools zoals Visual Studio Code of JetBrains IDE's.

    \item \textbf{(deelvraag 3)} Prestaties
    \item Een gedetailleerde vergelijking van de geselecteerde modellen met betrekking tot:
    \begin{itemize}
        \item Codegeneratienauwkeurigheid: Mate waarin de gegenereerde code correct en functioneel is, gemeten aan de hand van vooraf gedefinieerde testcases.
        \item Responstijd: De gemiddelde tijd (in seconden) om een programmeerprobleem op te lossen, gemeten voor taken van verschillende complexiteitsniveaus.
        \item Resourcegebruik: Het GPU-geheugen- en CPU-verbruik tijdens het genereren van code, vastgelegd in kwantitatieve metingen.
    \end{itemize}
    \item Deze resultaten worden ondersteund door gevisualiseerde data, zoals:
    \begin{itemize}
        \item Een staafdiagram voor nauwkeurigheid per model per taak.
        \item Een lijngrafiek voor responstijd in relatie tot taakcomplexiteit.
        \item Een spreidingsdiagram voor resourcegebruik tegenover modelprestaties.
    \end{itemize}
    \item Daarnaast wordt er aandacht besteed aan veiligheid: bevestiging dat de zelfgehoste modellen geen data opslaan of externe verbindingen leggen, wat cruciaal is voor het minimaliseren van datalekrisico’s.
    
    \item \textbf{(deelvraag 4)} Integratiegemak
    \item Een beoordeling van hoe eenvoudig deze modellen kunnen worden geïntegreerd in een typische ontwikkelworkflow. Hierbij wordt beschreven:
    \begin{itemize}
        \item De compatibiliteit met ontwikkelomgevingen (zoals Visual Studio Code of JetBrains IDE’s).
        \item De benodigde configuratiestappen, tools en plug-ins.
        \item De invloed op ontwikkelsnelheid en codekwaliteit vanuit het perspectief van een eindgebruiker.
    \item Deze resultaten bieden organisaties een gefundeerd overzicht van de voor- en nadelen van self-hosted LLMs, en dienen als leidraad bij de selectie en integratie van dergelijke modellen in hun eigen ontwikkelpraktijk.
    \end{itemize}
\end{itemize}

\subsection{Conclusie}

De verwachting is dat StarCoder beter presteert op complexere programmeerproblemen vanwege de focus op codeertaken in de training, terwijl Mistral meer geschikt blijkt voor resource-efficiënte omgevingen door zijn kleinere modelgrootte en lagere hardwarevereisten. Beide modellen worden als geschikte kandidaten gezien voor specifieke bedrijfscontexten, afhankelijk van de prioriteit op nauwkeurigheid of kosten.

De meerwaarde van dit onderzoek ligt in het bieden van een concrete handleiding aan bedrijven die op zoek zijn naar veilige en efficiënte AI-oplossingen voor programmeerhulp. Dit helpt organisaties bij het maken van een weloverwogen keuze en ondersteunt de adoptie van self-hosted LLMs zonder datalekrisico’s. Hierdoor kunnen bedrijven innovatie en dataveiligheid combineren in hun ontwikkelprocessen. Eventuele afwijkingen van de hypothese, zoals onverwacht betere prestaties van een model of integratieproblemen, zullen nader worden onderzocht om een dieper inzicht te krijgen in de praktische toepasbaarheid van deze technologie.

%%---------- Andere bijlagen --------------------------------------------------
% TODO: Voeg hier eventuele andere bijlagen toe. Bv. als je deze BP voor de
% tweede keer indient, een overzicht van de verbeteringen t.o.v. het origineel.
%\input{...}

%%---------- Backmatter, referentielijst ---------------------------------------

\backmatter{}

\setlength\bibitemsep{2pt} %% Add Some space between the bibliograpy entries
\printbibliography[heading=bibintoc]

\cleardoublepage
\phantomsection
\addcontentsline{toc}{chapter}{Glossary}
\printglossary[type=\acronymtype, title=Glossary, toctitle=Acronyms]

\end{document}
