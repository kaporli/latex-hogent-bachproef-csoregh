%%=============================================================================
%% Voorwoord
%%=============================================================================
\chapter*{\IfLanguageName{dutch}{Woord vooraf}{Preface}}%
\label{ch:voorwoord}

%% TODO:
%% Het voorwoord is het enige deel van de bachelorproef waar je vanuit je
%% eigen standpunt (``ik-vorm'') mag schrijven. Je kan hier bv. motiveren
%% waarom jij het onderwerp wil bespreken.
%% Vergeet ook niet te bedanken wie je geholpen/gesteund/... heeft
The inspiration for this thesis arose from an internship I completed at StadimData, a company focused on real estate data analytics. While there, I was explicitly instructed not to submit any proprietary code to external AI services such as GitHub Copilot or ChatGPT, due to the risk of exposing sensitive business information. This constraint prompted me to explore whether comparable AI‐driven coding support could be achieved entirely on local hardware, without depending on cloud platforms.

Because I needed a thesis topic I could design, implement, and evaluate independently, I settled on an empirical investigation of open‐weight code‐generation models operating on consumer‐grade workstations. This study weaves together system implementation, rigorous benchmarking, and enterprise security requirements into a unified research effort.

I owe particular thanks to Antoine Leflon, whose guidance was invaluable in defining the enterprise constraints and shaping the model‐selection criteria that underpin this work. Without his insight into real‐world deployment needs, I would have struggled to frame a relevant and practical evaluation.

\noindent Elias Alexander Csoregh\\
Vienna, May 2025
